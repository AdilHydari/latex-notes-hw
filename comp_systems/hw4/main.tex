\documentclass[12pt]{article}

\usepackage[utf8]{inputenc}
\usepackage{amsmath, amssymb}
\usepackage{geometry}
\usepackage{graphicx}
\usepackage{enumerate}
\usepackage{listings}
\usepackage{xcolor}
\usepackage{booktabs} 
\usepackage{hyperref}
\usepackage[utf8]{inputenc}
\usepackage{amsmath}
\begin{document}
	\title{Homework 4 - Computer Systems}
	\author{Adil Hydari}
	\date{}
	\maketitle
	
	% Question 1
	\section*{Q1: Signal Masking in Parent Process}
	\subsection*{1.1 Normal Execution}
	The parent process starts by constructing a signal mask that excludes four specific signals. By blocking \textbf{SIGTSTP} (terminal stop), \textbf{SIGTERM} (termination request), \textbf{SIGINT} (keyboard interrupt), and \textbf{SIGFPE} (arithmetic error), it ensures that any critical sections of code are not interrupted. This mask applies from process start to the first explicit handler invocation.
	
	\subsection*{1.2 Handler on SIGFPE, SIGHUP, SIGCHLD}
	When \texttt{sig\_handler} is invoked for \textbf{SIGFPE}, \textbf{SIGHUP} (hangup), or \textbf{SIGCHLD} (child status change), it inherits the existing mask. This prevents re-entry or interruption by those same signals while the handler runs, ensuring a consistent recovery procedure.
	
	\subsection*{1.3 Handler on SIGILL or SIGQUIT}
	If an illegal instruction (\textbf{SIGILL}) or quit request (\textbf{SIGQUIT}) arrives, the handler temporarily adds \textbf{SIGABRT} (abort) and \textbf{SIGQUIT} itself to the mask, on top of the original four. This extended mask protects the handler from nested aborts or quits, allowing it to log diagnostics or perform cleanup safely.
	
	\subsection*{1.4 Handler on SIGUSR1}
	A separate handler \texttt{sig\_usr1} addresses \textbf{SIGUSR1}. During its execution, the mask is extended to block \textbf{SIGABRT} and retain \textbf{SIGTERM}, plus the four standard signals. This guarantees that user-defined operations complete without unexpected termination.
	
	% Question 2
	\section*{Q2: Signal Handling in Parent Process}
	\subsection*{2.1 \texttt{sig\_handler} via \texttt{sigaction}}
	Using \texttt{sigaction}, the parent sets a persistent handler for:
	\begin{itemize}
		\item \textbf{SIGILL}: catches illegal instruction faults.
		\item \textbf{SIGINT}: handles keyboard interrupts when unmasked.
		\item \textbf{SIGQUIT}: intercepts user-initiated quits.
	\end{itemize}
	This handler remains installed until explicitly changed, ensuring consistent behavior on each arrival.
	
	\subsection*{2.2 \texttt{sig\_handler} via \texttt{signal} (one-shot)}
	A one-shot installation via \texttt{signal} is used for:
	\begin{itemize}
		\item \textbf{SIGFPE}: arithmetic exceptions.
		\item \textbf{SIGHUP}: hangup notifications.
		\item \textbf{SIGTSTP}: terminal stops.
		\item \textbf{SIGCHLD}: child termination or stop.
	\end{itemize}
	After handling one occurrence, the disposition resets to default, allowing subsequent signals to use their default action.
	
	\subsection*{2.3 \texttt{sig\_usr1} via \texttt{sigaction}}
	A persistent handler is installed for \textbf{SIGUSR1}, enabling the process to perform custom work on every delivery of the user-defined signal.
	
	% Question 3
	\section*{Q3: Signal Masking in Child Process}
	\subsection*{3.1 Normal Execution}
	The child process begins by blocking five signals to guard initialization:
	\begin{itemize}
		\item \textbf{SIGABRT}: aborts.
		\item \textbf{SIGTSTP}: stops.
		\item \textbf{SIGTERM}: terminations.
		\item \textbf{SIGSEGV}: segmentation faults.
		\item \textbf{SIGILL}: illegal instructions.
	\end{itemize}
	This mask ensures safe setup and shared-resource protection.
	
	\subsection*{3.2 Handler on SIGFPE, SIGHUP, SIGURG}
	When handling \textbf{SIGFPE}, \textbf{SIGHUP}, or \textbf{SIGURG} (urgent I/O), the handler retains the full five-signal mask, avoiding nested faults.
	
	\subsection*{3.3 Handler on SIGQUIT, SIGILL, SIGINT, SIGTSTP}
	Invoking \texttt{sig\_handler} for these signals adds \textbf{SIGABRT} and reaffirms \textbf{SIGTSTP}, while continuing to block \textbf{SIGTERM}, \textbf{SIGSEGV}, and \textbf{SIGILL}. This combination prevents simultaneous control-flow interrupts.
	
	\subsection*{3.4 Handler on SIGUSR1}
	The \texttt{sig\_usr1} handler in the child blocks \textbf{SIGABRT}, \textbf{SIGTSTP}, and \textbf{SIGTERM}. It also keeps \textbf{SIGSEGV} and \textbf{SIGILL} masked, ensuring critical cleanup can complete.
	
	% Question 4
	\section*{Q4: Actions from Signals (Lines 150--156, 183 and 157, 180--188)}
	The code explicitly sends signals on certain lines; here’s what happens.
	
	\subsection*{4.1 Sent to Child}
	\begin{itemize}
		\item Line 150 (\textbf{SIGUSR1}): delivered immediately to \texttt{sig\_usr1} via \texttt{sigaction}.
		\item Line 151 (\textbf{SIGQUIT}): caught by \texttt{sig\_handler} (persistent).
		\item Line 152 (\textbf{SIGILL}): currently masked, so remains pending.
		\item Line 153 (\textbf{SIGURG}): handled once by \texttt{sig\_handler} installed via \texttt{signal}.
		\item Line 154 (\textbf{SIGINT}): caught by persistent \texttt{sig\_handler}.
		\item Line 155 (\textbf{SIGSEGV}): blocked, pending delivery.
		\item Line 156 (\textbf{SIGTSTP}): masked by all handler masks, pending.
		\item Line 183 (\textbf{SIGFPE}): handled once; subsequent SIGFPE defaults after handler removal.
	\end{itemize}
	
	\subsection*{4.2 Sent to Parent}
	\begin{itemize}
		\item Line 157 (\textbf{SIGFPE}): currently masked, so pending.
		\item Line 180 (\textbf{SIGUSR1}): not sent (child’s PID=0).
		\item Line 181 (\textbf{SIGQUIT}): caught by \texttt{sig\_handler} (persistent).
		\item Line 182 (\textbf{SIGCONT}): default action resumes the process.
		\item Line 184 (\textbf{SIGILL}): delivered to \texttt{sig\_handler}.
		\item Line 185 (\textbf{SIGINT}): masked, pending.
		\item Line 186 (\textbf{SIGCHLD}): handled once by one-shot \texttt{sig\_handler}.
		\item Line 187 (\textbf{SIGHUP}): handled once, then resets.
		\item Line 188 (\textbf{SIGTSTP}): masked, pending.
	\end{itemize}
	
	% Question 5
	\section*{Q5: Scenario 2 (All Signals Sent)}
	Now assume both parent and child exchange every signal from lines 150--165 and 180--197.
	
	\subsection*{5.1 Child Receive Actions}
	Repeated deliveries mirror Q4. Key differences:
	\begin{itemize}
		\item Second SIGURG (line 158) arrives after removal of one-shot handler and is ignored.
		\item Subsequent SIGFPE, SIGHUP, SIGCHLD behave by default (terminate or ignore) once the one-shot handler is gone.
	\end{itemize}
	
	\subsection*{5.2 Parent Receive Actions}
	Follows Q4 with:
	\begin{itemize}
		\item Re-delivery of SIGHUP (line 164) and SIGFPE (line 165) now cause default termination.
		\item SIGCHLD at line 193 uses default ignore.
		\item Persistent handlers continue catching SIGQUIT and SIGILL.
	\end{itemize}
	
	\subsection*{5.3 Pending Signals}
	After all sends complete:
	\begin{itemize}
		\item \textbf{Parent:} SIGFPE, SIGINT, SIGTSTP remain in the waiting mask.
		\item \textbf{Child:} SIGILL, SIGSEGV, SIGTSTP remain pending.
	\end{itemize}
	
	% Question 6
	\section*{Q6: Terminating Signals}
	Any signal with default disposition \textit{terminate} or \textit{core dump} can kill the process when unhandled or after a one-shot handler is removed. Notable examples in this program include:\
	SIGKILL, SIGBUS, SIGTRAP, SIGSYS, SIGIOT; plus repeated SIGFPE or SIGHUP once their custom handlers have executed.
	
	% Question 7
	\section*{Q7: Threads and Signals (Problem 4B)}
	\subsection*{7.1 Q7.1: First Thread into \texttt{func}}
	Thread scheduling is non-deterministic. Regardless of which thread starts first,
	\begin{enumerate}
		\item The initial call to \texttt{func} computes denominator  safely.
		\item The second call sees , so denominator zero, raising \textbf{SIGFPE}. No handler exists yet, so that thread terminates.
		\item The third call uses denominator one and completes normally.
	\end{enumerate}
	
	\subsection*{7.2 Q7.2: Part /* B */ Lines}
	\paragraph{Line C (SIGSEGV to tid2, 3 times)}
	Thread 2 has a persistent handler for \textbf{SIGSEGV} via \texttt{sigaction}. Each of the three signals invokes the handler, producing three lines of output: ``Caught signal no = 11''.
	
	\paragraph{Line D (SIGINT to tid1, 3 times)}
	Thread 1 inherited \texttt{SIG\_IGN} for SIGINT before creation. Each delivered SIGINT is ignored, yielding no output.
	
	\subsection*{7.3 Q7.3: External \texttt{kill(pid,sig)} Signals}
	\paragraph{Q7.3.1: SIGSEGV (11)}
	Any thread not currently blocking SIGSEGV may handle it. The OS picks the first available; that thread’s re-installing handler prints ``Caught signal no = 11'' on each of the three kills.
	
	\paragraph{Q7.3.2: SIGQUIT (3)}
	Threads install \texttt{sig\_func} or \texttt{sig\_func2} at different times:
	\begin{itemize}
		\item If a thread without a custom handler receives SIGQUIT, default disposition terminates that thread.
		\item If tid3 or main (after its handler set) receives it, they print ``Caught signal no = 3'' and continue.
	\end{itemize}
	
	\paragraph{Q7.3.3: SIGBUS (7)}
	The first installation uses \texttt{sig\_func}, then overrides with \texttt{sig\_func2}. Each delivery invokes the active handler, printing ``Caught signal no = 7''.
	
	% Question 8
	\section*{Q8: Multi‐threaded Sieve (Problem 5A)}
	Implement two threads:
	\begin{itemize}
		\item \textbf{Sieve thread:} Allocate array , mark every multiple of each prime  (starting at ) as composite, stopping at . This yields all primes in  time.
		\item \textbf{Reversible-prime thread:} Iterate through marked primes. For each prime , reverse its decimal digits to get ; check . If  is also prime, print .
	\end{itemize}
	Threads share read-only and write-only regions and synchronize only on startup and shutdown.
	
	% Question 9
	\section*{Q9: Fibonacci Threads (Problem 5B)}
	Original solution:
	\begin{itemize}
		\item Main thread reads .
		\item Worker thread fills shared array .
		\item Main thread \texttt{pthread\_join}s, then prints all values.
	\end{itemize}
	
	To print values incrementally as they arrive:
	\begin{enumerate}
		\item Protect the shared array and an index counter with a mutex.
		\item Worker, after computing , locks the mutex, increments a count, signals a condition variable, then unlocks.
		\item Main, in a loop for  to , locks the mutex, waits on the condition until count>i, prints , unlocks.
	\end{enumerate}
	This allows the parent to output each Fibonacci number immediately when it becomes available.
	
\end{document}

