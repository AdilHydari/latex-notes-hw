\documentclass[12pt]{article}
\usepackage{amsmath,amssymb}
\usepackage{listings}
\usepackage{fullpage}
\usepackage{hyperref}
\usepackage{graphicx}
\usepackage{fancyhdr}

%\pagestyle{fancy}
%\fancyhead{}
%\fancyfoot{}
%\fancyhead[C]{ECE 434: Intro to Computer Systems -- Spring 2025}
%\fancyfoot[C]{\thepage}

\begin{document}
	
	% Title Page
	\begin{titlepage}
		\centering
		{\Large\bfseries ECE 434: Intro to Computer Systems -- Spring 2025 \par}
		{\large\textit{Adil Hydari} \par}
		\vspace{0.5cm}
		{\Large Homework 1 \par}
		\vspace{0.25cm}
		{\large Issue Date: Wednesday 02-19-2025 \par}
		{\large Due Date: Friday 02-28-2025 at 11:59 pm \par}
	\end{titlepage}
	
	\tableofcontents
	\newpage
	
	%%%%%%%%%%%%%%%%%%%%%%%%%%%%%%%%%%%%%%%
	% PROBLEM 1
	%%%%%%%%%%%%%%%%%%%%%%%%%%%%%%%%%%%%%%%
	\section{Problem 1: Interrupts (3 pts)}
	\textbf{Description:} \\
	Describe what a hardware interrupt, a trap, and an exception is, and list and justify as many differences as you can identify between any two of those.
	
	\subsection*{Answer}
	\begin{itemize}
		\item \textbf{Hardware Interrupt:} A signal from external hardware (e.g., an I/O device) to the CPU indicating that it needs immediate attention. Hardware interrupts are \emph{asynchronous} to program execution. When they occur, the CPU suspends its current instruction flow after finishing the current instruction and jumps to the appropriate interrupt service routine (ISR).
		\item \textbf{Trap:} A software-initiated interrupt. Typically triggered by a special instruction (e.g., a system call or a deliberate “breakpoint”). Traps are \emph{synchronous} events because they occur at a precise point during instruction execution (the point at which the program issues the instruction).
		\item \textbf{Exception:} A synchronous event caused by the CPU detecting an error/exception condition while executing an instruction (e.g., division by zero, invalid memory access, page fault). Exceptions cause control to transfer to an exception handler, which decides how to recover or terminate.
	\end{itemize}
	
	\textbf{Differences (example pairwise comparisons):}
	\begin{enumerate}
		\item \underline{Hardware Interrupt vs. Trap}:
		\begin{itemize}
			\item \textbf{Source}: Hardware interrupts come from external devices; traps are requested by software.
			\item \textbf{Timing}: Hardware interrupts are asynchronous; traps are synchronous with the instruction stream.
		\end{itemize}
		\item \underline{Trap vs. Exception}:
		\begin{itemize}
			\item \textbf{Intent}: A trap is generally an intentional software event (e.g., a system call), while an exception is often unintentional.
			\item \textbf{Handling}: Traps often pass control to well-defined OS routines, whereas exceptions might also raise error signals and can terminate or fix the failing instruction.
		\end{itemize}
	\end{enumerate}
	
	\hrule
	\vspace{0.5cm}
	
	%%%%%%%%%%%%%%%%%%%%%%%%%%%%%%%%%%%%%%%
	% PROBLEM 2
	%%%%%%%%%%%%%%%%%%%%%%%%%%%%%%%%%%%%%%%
	\section{Problem 2: Direct Memory Access (DMA) (6 pts)}
	\subsection*{Q1 (2 pts)}
	\textbf{Question:} \\
	Assume that a Network Card has data to transfer with the end goal to save the data in DRAM. Describe in detail the steps that need to be taken by the CPU, the Network Card, the user program, and the OS to coordinate the transfer (type of interrupts, register modifications, mode shifts, etc.).
	
	\subsubsection*{Answer}
	\begin{enumerate}
		\item The \textbf{user program} initiates a receive operation (e.g., via \texttt{recv()} system call).
		\item The \textbf{OS kernel} is invoked. In kernel mode, it allocates or identifies a DRAM buffer to store the incoming data.
		\item The OS programs the \textbf{DMA controller} by writing:
		\begin{itemize}
			\item The starting (base) address in DRAM.
			\item The size (bound) or length of the buffer.
			\item Possibly other control bits for the DMA engine.
		\end{itemize}
		\item The \textbf{Network Card} then takes over the actual transfer via DMA hardware. The CPU is free to context-switch to other tasks.
		\item Once the DMA completes, the Network Card raises a \textbf{DMA completion interrupt}.
		\item The CPU receives the interrupt, switches to \textbf{interrupt handler} in kernel mode, and checks the status registers.
		\item The OS marks the data buffer as “ready,” and then the user process is unblocked or notified that the data has arrived.
	\end{enumerate}
	
	\subsection*{Q2 (2 pts)}
	\textbf{Question:} How does the CPU know when the memory operations are complete?
	
	\subsubsection*{Answer}
	The device sends a \textbf{hardware interrupt} to signal completion. The interrupt service routine (ISR) in the OS then processes the completion event (e.g., by marking the data buffer valid).
	
	\subsection*{Q3 (2 pts)}
	\textbf{Question:} The CPU is allowed to execute other programs while the DMA controller is transferring data. Does this process interfere at any point during its life-cycle with the execution of user programs? If so, describe when and how interference(s) are caused.
	
	\subsubsection*{Answer}
	Yes, there are two points:
	\begin{enumerate}
		\item \textbf{Bus Contention}: The DMA controller uses the system bus to read/write memory. When the bus is busy for DMA, the CPU may experience slight performance delays if it also needs the bus.
		\item \textbf{Interrupt Handling}: When the DMA completes, the device raises an interrupt, forcing the CPU to switch to kernel mode to service it. This interrupt handling momentarily stalls user programs.
	\end{enumerate}
	
	\hrule
	\vspace{0.5cm}
	
	%%%%%%%%%%%%%%%%%%%%%%%%%%%%%%%%%%%%%%%
	% PROBLEM 3
	%%%%%%%%%%%%%%%%%%%%%%%%%%%%%%%%%%%%%%%
	\section{Problem 3: Familiarizing with Linux (6 pts)}
	\begin{enumerate}
		\item \textbf{(1 pt) Which instruction gives you your default shell?} \\
		\textit{Answer:}
		\begin{lstlisting}
			echo $SHELL
		\end{lstlisting}
		
		\item \textbf{(1 pt) Which instruction gives you the current shell?} \\
		\textit{Answer:}
		\begin{lstlisting}
			echo $0
		\end{lstlisting}
		\item \textbf{(1 pt) Which instruction gives you the process ID (PID) of your current shell?} \\
		\textit{Answer:}
		\begin{lstlisting}
			echo $$
		\end{lstlisting}
		
		\item \textbf{(1 pt) Which instructions do you need to use to switch from your default shell to another shell and back to your default shell?}\\
		\textit{Answer:} To switch shells, just type the name of the shell, e.g., \texttt{zsh} or \texttt{bash}. To go back, type \texttt{exit}.
		
		\item \textbf{(2 pts) Open two different shells in two different terminals, get information from both using \texttt{ls} (with a valid flag), \texttt{ps} (with a valid flag), or \texttt{htop}, then terminate the older shell from the newer shell. Show a screenshot.}\\
		\textit{Answer (example steps):}
		\begin{enumerate}
			\item Terminal 1: run \texttt{echo \$\$} to see the PID (5678).
			\item Terminal 2: run \texttt{ps -a} or \texttt{ps -ux} to confirm shell’s PID is 5678 in Terminal 1.
			\item In Terminal 2, type \texttt{kill 5678}.
			\item Terminal 1 shell is terminated.
		\end{enumerate}
		Screenshots would be attached here.
	\end{enumerate}
	
	\hrule
	\vspace{0.5cm}
	
	%%%%%%%%%%%%%%%%%%%%%%%%%%%%%%%%%%%%%%%
	% PROBLEM 4
	%%%%%%%%%%%%%%%%%%%%%%%%%%%%%%%%%%%%%%%
	\section{Problem 4: Altering the Simple Instruction Cycle (3 pts)}
	\textbf{Description:} \\
	We have user space addresses $> 100$ and kernel space addresses $\leq 100$. The \texttt{mode} register is 0 in kernel mode; 1 in user mode. Modify instruction fetch--decode--execute to handle memory violations or a timer interrupt. Then explain how the OS handles simultaneous events.
	
	\subsection*{Q1}
	\textbf{Question:} Which states/phases of handling an instruction (fetch--decode--execute) must be modified? Provide pseudo-code.
	
	\subsubsection*{Answer}
	We need to check for:
	\begin{itemize}
		\item \textbf{Memory violation}: If the CPU (in user mode) tries to access an address $\leq 100$, or an invalid address.
		\item \textbf{Timer interrupt}: If the timer expires.
	\end{itemize}
	
	\begin{verbatim}
		Pseudo-code:
		
		FETCH:
		IR <- MEM[PC]
		if (violation_detected or timer_interrupt_pending) then
		goto HANDLE_INTERRUPT
		PC <- PC + 4
		
		DECODE:
		decode IR
		if (timer_interrupt_pending) then
		goto HANDLE_INTERRUPT
		
		EXECUTE:
		execute instruction in IR
		if (memory_violation or timer_interrupt_pending) then
		goto HANDLE_INTERRUPT
		
		HANDLE_INTERRUPT:
		save_current_context()
		if memory_violation then
		// invoke memory violation
		// kill the user process or handle exception
		else if timer_interrupt_pending then
		// run timer handler
		restore_context()
	\end{verbatim}
	
	\subsection*{Q2}
	\textbf{Question:} What if the two events occur almost simultaneously?
	
	\subsubsection*{Answer}
	In most systems, interrupts have priority levels. For instance, a memory violation might have higher priority than a timer interrupt. The CPU or interrupt controller decides which interrupt to service first based on priority. If the timer interrupt is pending while a memory violation occurs, the memory-violation handler is invoked first. After handling the higher-priority event, the CPU can check if other interrupts are still pending.
	
	\hrule
	\vspace{0.5cm}
	
	%%%%%%%%%%%%%%%%%%%%%%%%%%%%%%%%%%%%%%%
	% PROBLEM 5
	%%%%%%%%%%%%%%%%%%%%%%%%%%%%%%%%%%%%%%%
	\section{Problem 5: DMA Quantitative Problem (6 pts)}
	\subsection*{Part (a) (2 pts)}
	\textbf{Question:} In the Polled I/O, why did we need to modify the Naïve Receive (RX) to Polled Receive but it would be ok to leave the Transmit (TX) to execute a while loop?
	
	\subsubsection*{Answer}
	\begin{itemize}
		\item \textbf{Naïve RX:} Continuously spinning (busy-waiting) for incoming data wastes CPU cycles. Polled receive only checks status when needed, or waits until an interrupt or periodic poll.
		\item \textbf{TX in a while loop:} Transmit simply pushes data out. Once we load the transmit buffer, the hardware will handle it. We only wait (in a loop) for the hardware to be “ready.” Because we are not missing data if we wait to transmit, this is less of a problem performance-wise.
	\end{itemize}
	
	\subsection*{Part (b) (4 pts)}
	\textbf{Question:} We have:
	\begin{itemize}
		\item Load: 300 clock cycles
		\item Store: 250 clock cycles
		\item Interrupt latency (1st interrupt): 2,500 instructions @ 2 cycles each
		\item Interrupt latency (2nd interrupt): 800 instructions @ 2 cycles each
		\item Setup for DMA: 4 loads (for base and bound registers)
	\end{itemize}
	Give a numerical example of packet size for which PIO < DMA. Then assume we can transmit more than one byte at a time using a 64-bit register. Which packet sizes favor PIO over DMA?
	
	\subsubsection*{Answer (Sample Computation)}
	\begin{itemize}
		\item \textbf{DMA overhead:}
		\[
		\text{DMA setup} = 4 \times (300\ \text{cycles per load}) = 1200\ \text{cycles}
		\]
		\[
		\text{Interrupt overhead} = (2500 + 800)\ \text{instructions} \times 2\ \text{cycles each} 
		= 3300 \times 2 = 6600\ \text{cycles}
		\]
		\[
		\text{Total DMA overhead} = 1200 + 6600 = 7800\ \text{cycles}
		\]
		
		\item \textbf{PIO overhead (per byte)}: Each byte store = 250 cycles. Let $N$ = number of bytes.
		\[
		\text{PIO cost} = 250 \times N
		\]
		If $250N < 7800$, then PIO is faster. That implies:
		\[
		N < \frac{7800}{250} \approx 31.2
		\]
		So, for example, a 30-byte packet has:
		\[
		\text{PIO cost} = 30 \times 250 = 7500 < 7800
		\]
		Hence, 30 bytes is an example where PIO is faster.
		
		\item \textbf{64-bit memory-mapped register}:
		- 64 bits = 8 bytes at a time.
		- Then the cost per 8-byte store is still $\approx 250$ cycles (one store instruction).
		- For $N$ bytes, we need $\frac{N}{8}$ stores, so
		\[
		\text{PIO cost} = \left(\frac{N}{8}\right) \times 250
		\]
		Set that $< 7800$:
		\[
		\frac{N}{8} \times 250 < 7800 \quad\Rightarrow\quad N < \frac{7800 \times 8}{250} = 249.6
		\]
		So for packets up to around 249 bytes, PIO is still faster; above that, DMA overhead pays off.
	\end{itemize}
	
	\hrule
	\vspace{0.5cm}
	
	%%%%%%%%%%%%%%%%%%%%%%%%%%%%%%%%%%%%%%%
	% PROBLEM 6
	%%%%%%%%%%%%%%%%%%%%%%%%%%%%%%%%%%%%%%%
	\section{Problem 6: Modes of Data Transfer (9 pts)}
	\subsection*{Q1 (2 pts)}
	\textbf{Question:} Why did we need to modify the Naïve Receive (RX) to Polled Receive but leave the Transmit task (TX) to execute a while loop?
	
	\subsubsection*{Answer}
	It closely mirrors Problem 5(a):
	\begin{itemize}
		\item \textbf{RX (Receive)}: Spinning naively can waste CPU cycles checking for new data. Polled receive is more efficient because we only check when necessary or when scheduled.
		\item \textbf{TX (Transmit)}: It is less harmful to wait in a while loop for the transmitter to be ready before sending the next chunk, because there is no risk of “missing” inbound data.
	\end{itemize}
	
	\subsection*{Q2 (2 pts)}
	\textbf{Question:} In Programmed I/O (PIO), what are the benefits of using Memory-Mapped Registers?
	
	\subsubsection*{Answer}
	\begin{itemize}
		\item The CPU can use the same instructions (\texttt{load}/\texttt{store}) to communicate with devices rather than special I/O instructions.
		\item Simplifies CPU design and can make device driver code simpler, because device registers appear in the address space.
		\item Compiler and OS can treat device registers similarly to normal memory, which can ease portability.
	\end{itemize}
	
	\subsection*{Q3 (5 pts)}
	\textbf{Question:} Polled vs. DMA Data Communication
	\begin{enumerate}
		\item (2 pts) In a Polled system, when the Status Registers change, a trap to the OS occurs. How often or under what circumstances does this trap happen? Describe the process and hardware components involved when a trap takes place.
		\item (3 pts) Why are the DMA interrupts more expensive than the trap in the Polled system? Is the trap in the Polled RX actually an interrupt? If so, why is there such a big difference in clock cycles compared to DMA interrupts?
	\end{enumerate}
	
	\subsubsection*{Answer}
	\begin{enumerate}
		\item \underline{Frequency of Trap in Polled RX}:
		\begin{itemize}
			\item Typically, the CPU executes a loop (in the driver or OS) checking a device’s status register. Whenever it detects that the status bits changed (e.g., new data arrived), it issues a \emph{software trap} (or system call) into kernel mode to process that data. 
			\item Alternatively, some architectures arrange it so that changes in a memory-mapped status register cause an exception, but most often it is simply the polling code that triggers a trap-like request to handle data.
		\end{itemize}
		
		\item \underline{DMA Interrupts vs. Polled Traps}:
		\begin{itemize}
			\item \textbf{DMA has two expensive interrupts}: one to start/prepare the transfer, another to signal completion. Both involve full interrupt-service overhead: saving CPU state, switching to kernel mode, possibly flushing pipelines or TLB, etc.
			\item \textbf{Polled trap overhead} is often lower because the CPU “knows” it is about to transition to kernel mode (the polling code explicitly checks and then calls the service routine). This avoids some overhead compared to an asynchronous hardware interrupt, which may force a pipeline flush, context save, and interrupt vector lookup.
			\item So yes, the polled trap is effectively a form of interrupt, but is usually simpler or partially handled in software, incurring fewer cycles than a hardware-driven, asynchronous interrupt with full context switching overhead.
		\end{itemize}
	\end{enumerate}
	
	\hrule
	\vspace{0.5cm}
	
	%%%%%%%%%%%%%%%%%%%%%%%%%%%%%%%%%%%%%%%
	% PROBLEM 7
	%%%%%%%%%%%%%%%%%%%%%%%%%%%%%%%%%%%%%%%
	\section{Problem 7: C Programming -- Files (10 pts)}
	\textbf{Tasks:}
	\begin{enumerate}
		\item Complete a program \texttt{merge\_files} that merges two input files into a third file. The default output is \texttt{myfile.out}.
		\item Show sample runs, including \texttt{strace} output.
	\end{enumerate}
	
	\subsection*{Answer}
	
	\lstinputlisting[language=C]{merge_files.c}
	
	\subsubsection*{Makefile}
	\begin{lstlisting}
		CC = gcc
		CFLAGS = -Wall -fsanitize=address
		
		all:
		$(CC) $(CFLAGS) -o merge_files merge_files.c
		
		clean:
		rm -f merge_files
	\end{lstlisting}
	
	\subsubsection*{Sample Runs}
	\begin{verbatim}
		
		$ echo "Data for file_in_1 This is OS 434 Sp25," > A1
		$ echo "data for file_in_2 but also OS 579 Sp25!" > A2
		
		$ ./merge_files A1 A2
		Merged A1 and A2 into myfile.out
		$ cat myfile.out
		Data for file_in_1 This is OS 434 Sp25,
		data for file_in_2 but also OS 579 Sp25!
		
		$ ./merge_files A1 A2 A3
		Merged A1 and A2 into A3
		$ cat A3
		Data for file_in_1 This is OS 434 Sp25,
		data for file_in_2 but also OS 579 Sp25!
	\end{verbatim}
	
	\subsubsection*{\texttt{strace} Output}
	\begin{verbatim}
	execve("./merge_files", ["./merge_files"], 0x7ffd227a0cc0 /* 44 vars */) = 0
	brk(NULL)                               = 0x5ab066c8d000
	access("/etc/ld.so.preload", R_OK)      = -1 ENOENT (No such file or directory)
	openat(AT_FDCWD, "/etc/ld.so.cache", O_RDONLY|O_CLOEXEC) = 3
	fstat(3, {st_mode=S_IFREG|0644, st_size=259939, ...}) = 0
	mmap(NULL, 259939, PROT_READ, MAP_PRIVATE, 3, 0) = 0x708c4e429000
	close(3)                                = 0
	mmap(NULL, 8192, PROT_READ|PROT_WRITE, MAP_PRIVATE|MAP_ANONYMOUS, -1, 0) = 0x708c4e427000
	openat(AT_FDCWD, "/usr/lib/libasan.so.8", O_RDONLY|O_CLOEXEC) = 3
	read(3, "\177ELF\2\1\1\0\0\0\0\0\0\0\0\0\3\0>\0\1\0\0\0\0\0\0\0\0\0\0\0"..., 832) = 832
	fstat(3, {st_mode=S_IFREG|0755, st_size=12004152, ...}) = 0
	mmap(NULL, 7558504, PROT_READ, MAP_PRIVATE|MAP_DENYWRITE, 3, 0) = 0x708c4dc00000
	mmap(0x708c4dc32000, 1605632, PROT_READ|PROT_EXEC, MAP_PRIVATE|MAP_FIXED|MAP_DENYWRITE, 3, 0x32000) = 0x708c4dc32000
	mmap(0x708c4ddba000, 262144, PROT_READ, MAP_PRIVATE|MAP_FIXED|MAP_DENYWRITE, 3, 0x1ba000) = 0x708c4ddba000
	mmap(0x708c4ddfa000, 32768, PROT_READ|PROT_WRITE, MAP_PRIVATE|MAP_FIXED|MAP_DENYWRITE, 3, 0x1f9000) = 0x708c4ddfa000
	mmap(0x708c4de02000, 5453160, PROT_READ|PROT_WRITE, MAP_PRIVATE|MAP_FIXED|MAP_ANONYMOUS, -1, 0) = 0x708c4de02000
	close(3)                                = 0
	openat(AT_FDCWD, "/usr/lib/libc.so.6", O_RDONLY|O_CLOEXEC) = 3
	read(3, "\177ELF\2\1\1\3\0\0\0\0\0\0\0\0\3\0>\0\1\0\0\0Pz\2\0\0\0\0\0"..., 832) = 832
	pread64(3, "\6\0\0\0\4\0\0\0@\0\0\0\0\0\0\0@\0\0\0\0\0\0\0@\0\0\0\0\0\0\0"..., 840, 64) = 840
	fstat(3, {st_mode=S_IFREG|0755, st_size=2022936, ...}) = 0
	pread64(3, "\6\0\0\0\4\0\0\0@\0\0\0\0\0\0\0@\0\0\0\0\0\0\0@\0\0\0\0\0\0\0"..., 840, 64) = 840
	mmap(NULL, 2047160, PROT_READ, MAP_PRIVATE|MAP_DENYWRITE, 3, 0) = 0x708c4da0c000
	mmap(0x708c4da30000, 1503232, PROT_READ|PROT_EXEC, MAP_PRIVATE|MAP_FIXED|MAP_DENYWRITE, 3, 0x24000) = 0x708c4da30000
	mmap(0x708c4db9f000, 339968, PROT_READ, MAP_PRIVATE|MAP_FIXED|MAP_DENYWRITE, 3, 0x193000) = 0x708c4db9f000
	mmap(0x708c4dbf2000, 24576, PROT_READ|PROT_WRITE, MAP_PRIVATE|MAP_FIXED|MAP_DENYWRITE, 3, 0x1e5000) = 0x708c4dbf2000
	mmap(0x708c4dbf8000, 31928, PROT_READ|PROT_WRITE, MAP_PRIVATE|MAP_FIXED|MAP_ANONYMOUS, -1, 0) = 0x708c4dbf8000
	close(3)                                = 0
	openat(AT_FDCWD, "/usr/lib/../lib/glibc-hwcaps/x86-64-v4/libstdc++.so.6", O_RDONLY|O_CLOEXEC) = -1 ENOENT (No such file or directory)
	newfstatat(AT_FDCWD, "/usr/lib/../lib/glibc-hwcaps/x86-64-v4/", 0x7ffee2380810, 0) = -1 ENOENT (No such file or directory)
	openat(AT_FDCWD, "/usr/lib/../lib/glibc-hwcaps/x86-64-v3/libstdc++.so.6", O_RDONLY|O_CLOEXEC) = -1 ENOENT (No such file or directory)
	newfstatat(AT_FDCWD, "/usr/lib/../lib/glibc-hwcaps/x86-64-v3/", 0x7ffee2380810, 0) = -1 ENOENT (No such file or directory)
	openat(AT_FDCWD, "/usr/lib/../lib/glibc-hwcaps/x86-64-v2/libstdc++.so.6", O_RDONLY|O_CLOEXEC) = -1 ENOENT (No such file or directory)
	newfstatat(AT_FDCWD, "/usr/lib/../lib/glibc-hwcaps/x86-64-v2/", 0x7ffee2380810, 0) = -1 ENOENT (No such file or directory)
	openat(AT_FDCWD, "/usr/lib/../lib/libstdc++.so.6", O_RDONLY|O_CLOEXEC) = 3
	read(3, "\177ELF\2\1\1\3\0\0\0\0\0\0\0\0\3\0>\0\1\0\0\0\0\0\0\0\0\0\0\0"..., 832) = 832
	fstat(3, {st_mode=S_IFREG|0755, st_size=25830928, ...}) = 0
	mmap(NULL, 2928768, PROT_READ, MAP_PRIVATE|MAP_DENYWRITE, 3, 0) = 0x708c4d600000
	mmap(0x708c4d694000, 1638400, PROT_READ|PROT_EXEC, MAP_PRIVATE|MAP_FIXED|MAP_DENYWRITE, 3, 0x94000) = 0x708c4d694000
	mmap(0x708c4d824000, 614400, PROT_READ, MAP_PRIVATE|MAP_FIXED|MAP_DENYWRITE, 3, 0x224000) = 0x708c4d824000
	mmap(0x708c4d8ba000, 57344, PROT_READ|PROT_WRITE, MAP_PRIVATE|MAP_FIXED|MAP_DENYWRITE, 3, 0x2ba000) = 0x708c4d8ba000
	mmap(0x708c4d8c8000, 12416, PROT_READ|PROT_WRITE, MAP_PRIVATE|MAP_FIXED|MAP_ANONYMOUS, -1, 0) = 0x708c4d8c8000
	close(3)                                = 0
	openat(AT_FDCWD, "/usr/lib/../lib/libm.so.6", O_RDONLY|O_CLOEXEC) = 3
	read(3, "\177ELF\2\1\1\3\0\0\0\0\0\0\0\0\3\0>\0\1\0\0\0\0\0\0\0\0\0\0\0"..., 832) = 832
	fstat(3, {st_mode=S_IFREG|0755, st_size=854360, ...}) = 0
	mmap(NULL, 856392, PROT_READ, MAP_PRIVATE|MAP_DENYWRITE, 3, 0) = 0x708c4e355000
	mmap(0x708c4e364000, 544768, PROT_READ|PROT_EXEC, MAP_PRIVATE|MAP_FIXED|MAP_DENYWRITE, 3, 0xf000) = 0x708c4e364000
	mmap(0x708c4e3e9000, 245760, PROT_READ, MAP_PRIVATE|MAP_FIXED|MAP_DENYWRITE, 3, 0x94000) = 0x708c4e3e9000
	mmap(0x708c4e425000, 8192, PROT_READ|PROT_WRITE, MAP_PRIVATE|MAP_FIXED|MAP_DENYWRITE, 3, 0xcf000) = 0x708c4e425000
	close(3)                                = 0
	openat(AT_FDCWD, "/usr/lib/../lib/libgcc_s.so.1", O_RDONLY|O_CLOEXEC) = 3
	read(3, "\177ELF\2\1\1\0\0\0\0\0\0\0\0\0\3\0>\0\1\0\0\0\0\0\0\0\0\0\0\0"..., 832) = 832
	fstat(3, {st_mode=S_IFREG|0644, st_size=1006416, ...}) = 0
	mmap(NULL, 205288, PROT_READ, MAP_PRIVATE|MAP_DENYWRITE, 3, 0) = 0x708c4d9d9000
	mmap(0x708c4d9dd000, 167936, PROT_READ|PROT_EXEC, MAP_PRIVATE|MAP_FIXED|MAP_DENYWRITE, 3, 0x4000) = 0x708c4d9dd000
	mmap(0x708c4da06000, 16384, PROT_READ, MAP_PRIVATE|MAP_FIXED|MAP_DENYWRITE, 3, 0x2d000) = 0x708c4da06000
	mmap(0x708c4da0a000, 8192, PROT_READ|PROT_WRITE, MAP_PRIVATE|MAP_FIXED|MAP_DENYWRITE, 3, 0x31000) = 0x708c4da0a000
	close(3)                                = 0
	mmap(NULL, 8192, PROT_READ|PROT_WRITE, MAP_PRIVATE|MAP_ANONYMOUS, -1, 0) = 0x708c4e353000
	mmap(NULL, 12288, PROT_READ|PROT_WRITE, MAP_PRIVATE|MAP_ANONYMOUS, -1, 0) = 0x708c4e350000
	arch_prctl(ARCH_SET_FS, 0x708c4e3507c0) = 0
	set_tid_address(0x708c4e350a90)         = 11013
	set_robust_list(0x708c4e350aa0, 24)     = 0
	rseq(0x708c4e350680, 0x20, 0, 0x53053053) = 0
	mprotect(0x708c4dbf2000, 16384, PROT_READ) = 0
	mprotect(0x708c4da0a000, 4096, PROT_READ) = 0
	mprotect(0x708c4e425000, 4096, PROT_READ) = 0
	mprotect(0x708c4d8ba000, 53248, PROT_READ) = 0
	mprotect(0x708c4ddfa000, 20480, PROT_READ) = 0
	mprotect(0x5ab035b86000, 4096, PROT_READ) = 0
	mprotect(0x708c4e4a6000, 8192, PROT_READ) = 0
	prlimit64(0, RLIMIT_STACK, NULL, {rlim_cur=16384*1024, rlim_max=RLIM64_INFINITY}) = 0
	munmap(0x708c4e429000, 259939)          = 0
	readlinkat(AT_FDCWD, "/proc/self/exe", "/home/adilh/latex-notes-hw/comp_"..., 4096) = 55
	mmap(NULL, 4096, PROT_READ|PROT_WRITE, MAP_PRIVATE|MAP_ANONYMOUS, -1, 0) = 0x708c4e468000
	openat(AT_FDCWD, "/proc/self/cmdline", O_RDONLY) = 3
	read(3, "./merge_files\0", 4096)        = 14
	read(3, "", 4082)                       = 0
	close(3)                                = 0
	munmap(0x708c4e468000, 4096)            = 0
	mmap(NULL, 4096, PROT_READ|PROT_WRITE, MAP_PRIVATE|MAP_ANONYMOUS, -1, 0) = 0x708c4e468000
	openat(AT_FDCWD, "/proc/self/environ", O_RDONLY) = 3
	read(3, "LC_ADDRESS=en_US.UTF-8\0MOTD_SHOW"..., 4096) = 1133
	read(3, "", 2963)                       = 0
	close(3)                                = 0
	mmap(NULL, 65536, PROT_READ|PROT_WRITE, MAP_PRIVATE|MAP_ANONYMOUS, -1, 0) = 0x708c4e458000
	mmap(NULL, 3727360, PROT_READ|PROT_WRITE, MAP_PRIVATE|MAP_ANONYMOUS, -1, 0) = 0x708c4d272000
	mmap(NULL, 2097152, PROT_READ|PROT_WRITE, MAP_PRIVATE|MAP_ANONYMOUS, -1, 0) = 0x708c4d000000
	munmap(0x708c4d100000, 1048576)         = 0
	mmap(NULL, 4096, PROT_READ|PROT_WRITE, MAP_PRIVATE|MAP_ANONYMOUS, -1, 0) = 0x708c4e457000
	mmap(NULL, 2097152, PROT_READ|PROT_WRITE, MAP_PRIVATE|MAP_ANONYMOUS, -1, 0) = 0x708c4ce00000
	munmap(0x708c4cf00000, 1048576)         = 0
	mmap(NULL, 4096, PROT_READ|PROT_WRITE, MAP_PRIVATE|MAP_ANONYMOUS, -1, 0) = 0x708c4e456000
	mmap(NULL, 2097152, PROT_READ|PROT_WRITE, MAP_PRIVATE|MAP_ANONYMOUS, -1, 0) = 0x708c4cc00000
	munmap(0x708c4cd00000, 1048576)         = 0
	mmap(NULL, 2097152, PROT_READ|PROT_WRITE, MAP_PRIVATE|MAP_ANONYMOUS, -1, 0) = 0x708c4ca00000
	munmap(0x708c4cb00000, 1048576)         = 0
	mmap(NULL, 2097152, PROT_READ|PROT_WRITE, MAP_PRIVATE|MAP_ANONYMOUS, -1, 0) = 0x708c4c800000
	munmap(0x708c4c900000, 1048576)         = 0
	mmap(NULL, 4096, PROT_READ|PROT_WRITE, MAP_PRIVATE|MAP_ANONYMOUS, -1, 0) = 0x708c4e455000
	prlimit64(0, RLIMIT_CORE, NULL, {rlim_cur=RLIM64_INFINITY, rlim_max=RLIM64_INFINITY}) = 0
	prlimit64(0, RLIMIT_CORE, {rlim_cur=0, rlim_max=RLIM64_INFINITY}, NULL) = 0
	mmap(NULL, 4096, PROT_READ|PROT_WRITE, MAP_PRIVATE|MAP_ANONYMOUS, -1, 0) = 0x708c4e454000
	openat(AT_FDCWD, "/proc/self/maps", O_RDONLY) = 3
	read(3, "5ab035b83000-5ab035b84000 r--p 0"..., 4096) = 4091
	read(3, "708c4", 5)                     = 5
	close(3)                                = 0
	munmap(0x708c4e454000, 4096)            = 0
	mmap(NULL, 8192, PROT_READ|PROT_WRITE, MAP_PRIVATE|MAP_ANONYMOUS, -1, 0) = 0x708c4e453000
	openat(AT_FDCWD, "/proc/self/maps", O_RDONLY) = 3
	read(3, "5ab035b83000-5ab035b84000 r--p 0"..., 8192) = 4091
	read(3, "708c4e49b000-708c4e4a6000 r--p 0"..., 4101) = 523
	read(3, "", 3578)                       = 0
	close(3)                                = 0
	mmap(NULL, 4096, PROT_READ|PROT_WRITE, MAP_PRIVATE|MAP_ANONYMOUS, -1, 0) = 0x708c4e452000
	openat(AT_FDCWD, "/proc/self/maps", O_RDONLY) = 3
	read(3, "5ab035b83000-5ab035b84000 r--p 0"..., 4096) = 4091
	read(3, "708c4", 5)                     = 5
	close(3)                                = 0
	munmap(0x708c4e452000, 4096)            = 0
	mmap(NULL, 8192, PROT_READ|PROT_WRITE, MAP_PRIVATE|MAP_ANONYMOUS, -1, 0) = 0x708c4e451000
	openat(AT_FDCWD, "/proc/self/maps", O_RDONLY) = 3
	read(3, "5ab035b83000-5ab035b84000 r--p 0"..., 8192) = 4091
	read(3, "708c4e49b000-708c4e4a6000 r--p 0"..., 4101) = 523
	read(3, "", 3578)                       = 0
	close(3)                                = 0
	munmap(0x708c4e451000, 8192)            = 0
	mmap(0x7fff7000, 268435456, PROT_READ|PROT_WRITE, MAP_PRIVATE|MAP_FIXED|MAP_ANONYMOUS|MAP_NORESERVE, -1, 0) = 0x7fff7000
	madvise(0x7fff7000, 268435456, MADV_NOHUGEPAGE) = 0
	madvise(0x7fff7000, 268435456, MADV_DONTDUMP) = 0
	mmap(0x2008fff7000, 15392894357504, PROT_READ|PROT_WRITE, MAP_PRIVATE|MAP_FIXED|MAP_ANONYMOUS|MAP_NORESERVE, -1, 0) = 0x2008fff7000
	madvise(0x2008fff7000, 15392894357504, MADV_NOHUGEPAGE) = 0
	madvise(0x2008fff7000, 15392894357504, MADV_DONTDUMP) = 0
	mmap(0x8fff7000, 2199023255552, PROT_NONE, MAP_PRIVATE|MAP_FIXED|MAP_ANONYMOUS|MAP_NORESERVE, -1, 0) = 0x8fff7000
	sigaltstack(NULL, {ss_sp=NULL, ss_flags=SS_DISABLE, ss_size=0}) = 0
	mmap(NULL, 57344, PROT_READ|PROT_WRITE, MAP_PRIVATE|MAP_ANONYMOUS, -1, 0) = 0x708c4e445000
	sigaltstack({ss_sp=0x708c4e445000, ss_flags=0, ss_size=54016}, NULL) = 0
	rt_sigaction(SIGSEGV, {sa_handler=0x708c4dd4b890, sa_mask=[], sa_flags=SA_RESTORER|SA_ONSTACK|SA_NODEFER|SA_SIGINFO, sa_restorer=0x708c4da4ea20}, NULL, 8) = 0
	rt_sigaction(SIGBUS, {sa_handler=0x708c4dd4b890, sa_mask=[], sa_flags=SA_RESTORER|SA_ONSTACK|SA_NODEFER|SA_SIGINFO, sa_restorer=0x708c4da4ea20}, NULL, 8) = 0
	rt_sigaction(SIGFPE, {sa_handler=0x708c4dd4b890, sa_mask=[], sa_flags=SA_RESTORER|SA_ONSTACK|SA_NODEFER|SA_SIGINFO, sa_restorer=0x708c4da4ea20}, NULL, 8) = 0
	mmap(0x500000000000, 4398046519296, PROT_NONE, MAP_PRIVATE|MAP_FIXED|MAP_ANONYMOUS|MAP_NORESERVE, -1, 0) = 0x500000000000
	mmap(0x540000000000, 8192, PROT_READ|PROT_WRITE, MAP_PRIVATE|MAP_FIXED|MAP_ANONYMOUS, -1, 0) = 0x540000000000
	mmap(NULL, 8388608, PROT_NONE, MAP_PRIVATE|MAP_ANONYMOUS|MAP_NORESERVE, -1, 0) = 0x708c4c000000
	mmap(NULL, 57344, PROT_READ|PROT_WRITE, MAP_PRIVATE|MAP_ANONYMOUS, -1, 0) = 0x708c4e437000
	mmap(NULL, 4096, PROT_READ|PROT_WRITE, MAP_PRIVATE|MAP_ANONYMOUS, -1, 0) = 0x708c4e436000
	getpid()                                = 11013
	prlimit64(0, RLIMIT_STACK, NULL, {rlim_cur=16384*1024, rlim_max=RLIM64_INFINITY}) = 0
	mmap(NULL, 4096, PROT_READ|PROT_WRITE, MAP_PRIVATE|MAP_ANONYMOUS, -1, 0) = 0x708c4e435000
	openat(AT_FDCWD, "/proc/self/maps", O_RDONLY) = 3
	read(3, "7fff7000-8fff7000 rw-p 00000000 "..., 4096) = 4085
	read(3, "708c4e46d00", 11)              = 11
	close(3)                                = 0
	munmap(0x708c4e435000, 4096)            = 0
	mmap(NULL, 8192, PROT_READ|PROT_WRITE, MAP_PRIVATE|MAP_ANONYMOUS, -1, 0) = 0x708c4e434000
	openat(AT_FDCWD, "/proc/self/maps", O_RDONLY) = 3
	read(3, "7fff7000-8fff7000 rw-p 00000000 "..., 8192) = 4085
	read(3, "708c4e46d000-708c4e46f000 r-xp 0"..., 4107) = 809
	read(3, "", 3298)                       = 0
	close(3)                                = 0
	munmap(0x708c4e453000, 8192)            = 0
	mmap(NULL, 4096, PROT_READ|PROT_WRITE, MAP_PRIVATE|MAP_ANONYMOUS, -1, 0) = 0x708c4e454000
	openat(AT_FDCWD, "/proc/self/maps", O_RDONLY) = 3
	read(3, "7fff7000-8fff7000 rw-p 00000000 "..., 4096) = 4047
	read(3, "708c4e46b000-708c4e46d000 r--p 0"..., 49) = 49
	close(3)                                = 0
	munmap(0x708c4e454000, 4096)            = 0
	mmap(NULL, 8192, PROT_READ|PROT_WRITE, MAP_PRIVATE|MAP_ANONYMOUS, -1, 0) = 0x708c4e453000
	openat(AT_FDCWD, "/proc/self/maps", O_RDONLY) = 3
	read(3, "7fff7000-8fff7000 rw-p 00000000 "..., 8192) = 4085
	read(3, "708c4e46d000-708c4e46f000 r-xp 0"..., 4107) = 809
	read(3, "", 3298)                       = 0
	close(3)                                = 0
	munmap(0x708c4e453000, 8192)            = 0
	mmap(0x10005c269000, 2093056, PROT_READ|PROT_WRITE, MAP_PRIVATE|MAP_FIXED|MAP_ANONYMOUS|MAP_NORESERVE, -1, 0) = 0x10005c269000
	madvise(0x10005c269000, 2093056, MADV_NOHUGEPAGE) = 0
	madvise(0x10005c269000, 2093056, MADV_DONTDUMP) = 0
	mmap(NULL, 11571200, PROT_READ|PROT_WRITE, MAP_PRIVATE|MAP_ANONYMOUS, -1, 0) = 0x708c4b4f7000
	mmap(NULL, 4096, PROT_READ|PROT_WRITE, MAP_PRIVATE|MAP_ANONYMOUS, -1, 0) = 0x708c4e454000
	sigaltstack(NULL, {ss_sp=0x708c4e445000, ss_flags=0, ss_size=54016}) = 0
	mmap(NULL, 1703936, PROT_READ|PROT_WRITE, MAP_PRIVATE|MAP_ANONYMOUS, -1, 0) = 0x708c4b357000
	mmap(NULL, 4096, PROT_READ|PROT_WRITE, MAP_PRIVATE|MAP_ANONYMOUS, -1, 0) = 0x708c4e453000
	munmap(0x708c4e453000, 4096)            = 0
	mmap(NULL, 2097152, PROT_READ|PROT_WRITE, MAP_PRIVATE|MAP_ANONYMOUS, -1, 0) = 0x708c4b000000
	munmap(0x708c4b100000, 1048576)         = 0
	mmap(NULL, 2097152, PROT_READ|PROT_WRITE, MAP_PRIVATE|MAP_ANONYMOUS, -1, 0) = 0x708c4ae00000
	munmap(0x708c4af00000, 1048576)         = 0
	mmap(NULL, 4096, PROT_READ|PROT_WRITE, MAP_PRIVATE|MAP_ANONYMOUS, -1, 0) = 0x708c4e453000
	mmap(NULL, 65536, PROT_READ|PROT_WRITE, MAP_PRIVATE|MAP_ANONYMOUS, -1, 0) = 0x708c4e340000
	mmap(NULL, 8192, PROT_READ|PROT_WRITE, MAP_PRIVATE|MAP_ANONYMOUS, -1, 0) = 0x708c4e432000
	clock_gettime(CLOCK_MONOTONIC, {tv_sec=1961, tv_nsec=978255191}) = 0
	mmap(0x506000000000, 65536, PROT_READ|PROT_WRITE, MAP_PRIVATE|MAP_FIXED|MAP_ANONYMOUS, -1, 0) = 0x506000000000
	mmap(0x506e00000000, 65536, PROT_READ|PROT_WRITE, MAP_PRIVATE|MAP_FIXED|MAP_ANONYMOUS, -1, 0) = 0x506e00000000
	mmap(NULL, 1048576, PROT_READ|PROT_WRITE, MAP_PRIVATE|MAP_ANONYMOUS, -1, 0) = 0x708c4d8d9000
	mmap(NULL, 8388608, PROT_READ|PROT_WRITE, MAP_PRIVATE|MAP_ANONYMOUS|MAP_NORESERVE, -1, 0) = 0x708c4a600000
	clock_gettime(CLOCK_MONOTONIC, {tv_sec=1961, tv_nsec=978850719}) = 0
	mmap(0x503000000000, 65536, PROT_READ|PROT_WRITE, MAP_PRIVATE|MAP_FIXED|MAP_ANONYMOUS, -1, 0) = 0x503000000000
	mmap(0x503e00000000, 65536, PROT_READ|PROT_WRITE, MAP_PRIVATE|MAP_FIXED|MAP_ANONYMOUS, -1, 0) = 0x503e00000000
	futex(0x708c4d8c86bc, FUTEX_WAKE_PRIVATE, 2147483647) = 0
	mmap(0x531000000000, 131072, PROT_READ|PROT_WRITE, MAP_PRIVATE|MAP_FIXED|MAP_ANONYMOUS, -1, 0) = 0x531000000000
	mmap(0x531e00000000, 65536, PROT_READ|PROT_WRITE, MAP_PRIVATE|MAP_FIXED|MAP_ANONYMOUS, -1, 0) = 0x531e00000000
	mmap(NULL, 4096, PROT_READ|PROT_WRITE, MAP_PRIVATE|MAP_ANONYMOUS, -1, 0) = 0x708c4e431000
	fstat(1, {st_mode=S_IFCHR|0620, st_rdev=makedev(0x88, 0), ...}) = 0
	mmap(0x519000000000, 65536, PROT_READ|PROT_WRITE, MAP_PRIVATE|MAP_FIXED|MAP_ANONYMOUS, -1, 0) = 0x519000000000
	mmap(0x519e00000000, 65536, PROT_READ|PROT_WRITE, MAP_PRIVATE|MAP_FIXED|MAP_ANONYMOUS, -1, 0) = 0x519e00000000
	write(1, "Usage: ./merge_files file_in_1 f"..., 73Usage: ./merge_files file_in_1 file_in_2 [file_out (default:myfile.out)]
	) = 73
	gettid()                                = 11013
	gettid()                                = 11013
	prctl(PR_GET_DUMPABLE)                  = 1 (SUID_DUMP_USER)
	getpid()                                = 11013
	mmap(NULL, 2101248, PROT_READ|PROT_WRITE, MAP_PRIVATE|MAP_ANONYMOUS, -1, 0) = 0x708c4b156000
	mprotect(0x708c4b156000, 4096, PROT_NONE) = 0
	rt_sigprocmask(SIG_BLOCK, ~[ILL ABRT BUS FPE SEGV XCPU XFSZ], [], 8) = 0
	clone(child_stack=0x708c4b356ff0, flags=CLONE_VM|CLONE_FS|CLONE_FILES|CLONE_UNTRACED) = 11014
	rt_sigprocmask(SIG_SETMASK, [], NULL, 8) = 0
	getpid()                                = 11013
	prctl(PR_SET_PTRACER, 11014)            = 0
	futex(0x7ffee23813b8, FUTEX_WAKE_PRIVATE, 1) = 1
	sched_yield()                           = 0
	sched_yield()                           = 0
	sched_yield()                           = 0
	sched_yield()                           = 0
	sched_yield()                           = 0
	sched_yield()                           = 0
	sched_yield()                           = 0
	sched_yield()                           = 0
	wait4(11014, NULL, __WALL, NULL)        = 11014
	munmap(0x708c4b156000, 2101248)         = 0
	getpid()                                = 11013
	write(2, "==11013==LeakSanitizer has encou"..., 54==11013==LeakSanitizer has encountered a fatal error.
	) = 54
	mmap(NULL, 65536, PROT_READ|PROT_WRITE, MAP_PRIVATE|MAP_ANONYMOUS, -1, 0) = 0x708c4d262000
	getpid()                                = 11013
	write(2, "==11013==HINT: For debugging, tr"..., 102==11013==HINT: For debugging, try setting environment variable LSAN_OPTIONS=verbosity=1:log_threads=1
	) = 102
	getpid()                                = 11013
	write(2, "==11013==HINT: LeakSanitizer doe"..., 75==11013==HINT: LeakSanitizer does not work under ptrace (strace, gdb, etc)
	) = 75
	exit_group(1)                           = ?
	+++ exited with 1 +++
	\end{verbatim}
	
	\hrule
	\vspace{0.5cm}
	
	%%%%%%%%%%%%%%%%%%%%%%%%%%%%%%%%%%%%%%%
	% PROBLEM 8
	%%%%%%%%%%%%%%%%%%%%%%%%%%%%%%%%%%%%%%%
	\section{Problem 8: Linux Shell Scripting (10 pts)}
	\textbf{Description:} \\
	Create 8 text files, each with a random number of lines. Write a shell script that:
	\begin{itemize}
		\item Takes a number of these files as arguments.
		\item Counts lines in each, adds them cumulatively, prints results to screen and a file, and shows total files processed.
	\end{itemize}
	
	\subsection*{Answer (\texttt{line\_counter.sh})}
	\lstinputlisting[language=bash]{line_counter.sh}
	
	\subsubsection*{How to Run}
	\begin{verbatim}
		1) Create 8 new .txt files (e.g., f1.txt, f2.txt, etc.) and fill them with random lines.
		2) Make script executable:  chmod +x line_counter.sh
		3) Run:  ./line_counter.sh f1.txt f2.txt ...
		4) Output is shown on screen and stored in line_count_results.txt
	\end{verbatim}
	
	\subsubsection*{Output}
	\begin{verbatim}
		~/latex-notes-hw/comp_systems/hw1 comp_systems
		- ./line_counter.sh A1 A2
		File: A1 has 1 lines.
		File: A2 has 1 lines.
		----------------------------------
		Number of files processed: 2
		Total number of lines: 2
	\end{verbatim}
	
\end{document}
