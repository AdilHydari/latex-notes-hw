%%%%%%%%%%%%%%%%%%%%%%%%%%%%%%%%%%%%%%%%%

% Chapter heading images should have a 2:1 width:height ratio,
% e.g. 920px width and 460px height.
%
%%%%%%%%%%%%%%%%%%%%%%%%%%%%%%%%%%%%%%%%%


%----------------------------------------------------------------------------------------
%	PACKAGES AND OTHER DOCUMENT CONFIGURATIONS
%----------------------------------------------------------------------------------------

\documentclass[11pt,fleqn]{book} % Default font size and left-justified equations

\usepackage[top=3cm,bottom=3cm,left=3.2cm,right=3.2cm,headsep=10pt,letterpaper]{geometry} % Page margins

\usepackage{xcolor} % Required for specifying colors by name
\definecolor{ocre}{RGB}{52,177,201} % Define the orange color used for highlighting throughout the book

% Font Settings
\usepackage{avant} % Use the Avantgarde font for headings
%\usepackage{times} % Use the Times font for headings
\usepackage{mathptmx} % Use the Adobe Times Roman as the default text font together with math symbols from the Symbol, Chancery and Computer Modern fonts
\usepackage{microtype} % Slightly tweak font spacing for aesthetics
\usepackage[utf8]{inputenc} % Required for including letters with accents
\usepackage[T1]{fontenc} % Use 8-bit encoding that has 256 glyphs
\usepackage{amsthm}

% Bibliography
\usepackage[style=alphabetic,sorting=nyt,sortcites=true,autopunct=true,babel=hyphen,hyperref=true,abbreviate=false,backref=true,backend=biber]{biblatex}
\addbibresource{bibliography.bib} % BibTeX bibliography file
\defbibheading{bibempty}{}

\input{structure} % Insert the commands.tex file which contains the majority of the structure behind the template

%----------------------------------------------------------------------------------------
%	Definitions of new commands
%----------------------------------------------------------------------------------------


\begin{document}

%----------------------------------------------------------------------------------------
%	TITLE PAGE
%----------------------------------------------------------------------------------------

\begingroup
\thispagestyle{empty}
\AddToShipoutPicture*{\put(0,0){\includegraphics[scale=1.25]{esahubble}}} % Image background
\centering
\vspace*{5cm}
\par\normalfont\fontsize{35}{35}\sffamily\selectfont
\textbf{Eastern Europe}\\
{\LARGE History of Eastern Europe from 1800-1948}\par % Book title
\vspace*{1cm}
{\Huge Lecture Notes}\par % Author name
\endgroup

%----------------------------------------------------------------------------------------
%	COPYRIGHT PAGE
%----------------------------------------------------------------------------------------

\newpage
~\vfill
\thispagestyle{empty}

%\noindent Copyright \copyright\ 2014 Andrea Hidalgo\\ % Copyright notice

 % URL

\\ % License information

\noindent \textit{Fall 2024} % Printing/edition date

%----------------------------------------------------------------------------------------
%	TABLE OF CONTENTS
%----------------------------------------------------------------------------------------

%\chapterimage{head1.png} % Table of contents heading image

\pagestyle{empty} % No headers

\tableofcontents % Print the table of contents itself

%\cleardoublepage % Forces the first chapter to start on an odd page so it's on the right

\pagestyle{fancy} % Print headers again

%----------------------------------------------------------------------------------------
%	CHAPTER 1
%----------------------------------------------------------------------------------------

%\chapterimage{head2.png} % Chapter heading image
\chapter{Week 2}
\section{Wolff - Inventing Eastern Europe}
\subsection{Pages 1-8}
\begin{definition}[]

\end{definition}

\begin{example}

\end{example}

\subsection{Pages 25-38}
\begin{itemize}
\item[Intersection] 
\item[Linear map] 
\item[Inverse image] 
\end{itemize}

\section{Magocsi - Ethnolinguistic Map }
\subsection{Pages 97-99}

\begin{example}

\end{example}

\begin{theorem}[]

\end{theorem}

\section{Nations and Nationalism As Concepts}

\subsection{Anderson Imagined Communities}

\subsection{Gellner Nations Nationalism}

\subsection{Judson Beyond Nations}

\section{Nationalism and Emancipation}
\subsection{Execution of Louis XVI}
\subsection{DECLARATION OF THE RIGHTS OF MAN AND CITIZEN}
\begin{itemize}
    \item  "I only say that (rights) cannot exist except as a consequence of duties fulfilled"
    \item "Each man looked after his own rights and the improvement of his own condition without seeking to provide for others" ... "(A war) in which those who had the means and were strong relentlessly crushed the weak or the unskilled."
\end{itemize}
\subsection{Napoleonic empire - 1812 }
The Napoleonic code: set of laws that replaced the laws where Napoleon's empire stretched. 
\subsection{Mazzini}
\begin{itemize}
    \item Italian Revolutionary, always crushed, time in jail (exile) 
    \item Wanted an unified Italian republic, this revolution would transform Italy and later Europe.
    \item If the condition of the populace has improved why then, "has the condition of the people not improved?".
\end{itemize}
\section{National Rebirth}
\subsection{Fichte Love of Fatherland}

\subsection{Arndt German Fatherland}
 
\end{document}