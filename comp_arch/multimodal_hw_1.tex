\documentclass[letterpaper, 11pt]{extarticle}
% \usepackage{fontspec}

% ==================================================

% document parameters
% \usepackage[spanish, mexico, es-lcroman]{babel}
\usepackage[english]{babel}
\usepackage[margin = 1in]{geometry}

% ==================================================

% Packages for math
\usepackage{mathrsfs}
\usepackage{amsfonts}
\usepackage{amsmath}
\usepackage{amsthm}
\usepackage{amssymb}
\usepackage{physics}
\usepackage{dsfont}
\usepackage{esint}

% ==================================================

% Packages for writing
\usepackage{enumerate}
\usepackage[shortlabels]{enumitem}
\usepackage{framed}
\usepackage{csquotes}

% ==================================================

% Miscellaneous packages
\usepackage{float}
\usepackage{tabularx}
\usepackage{xcolor}
\usepackage{multicol}
\usepackage{subcaption}
\usepackage{caption}
\captionsetup{format = hang, margin = 10pt, font = small, labelfont = bf}

% Citation
\usepackage[round, authoryear]{natbib}

% Hyperlinks setup
\usepackage{hyperref}
\definecolor{links}{rgb}{0.36,0.54,0.66}
\hypersetup{
	colorlinks = true,
	linkcolor = black,
	urlcolor = blue,
	citecolor = blue,
	filecolor = blue,
	pdfauthor = {Author},
	pdftitle = {Title},
	pdfsubject = {subject},
	pdfkeywords = {one, two},
	pdfproducer = {LaTeX},
	pdfcreator = {pdfLaTeX},
}
\usepackage{titlesec}
\usepackage[many]{tcolorbox}

% Adjust spacing after the chapter title
\titlespacing*{\chapter}{0cm}{-2.0cm}{0.50cm}
\titlespacing*{\section}{0cm}{0.50cm}{0.25cm}

% Indent 
\setlength{\parindent}{0pt}
\setlength{\parskip}{1ex}

% --- Theorems, lemma, corollary, postulate, definition ---
% \numberwithin{equation}{section}

\newtcbtheorem[]{problem}{Problem}%
{enhanced,
	colback = black!5, %white,
	colbacktitle = black!5,
	coltitle = black,
	boxrule = 0pt,
	frame hidden,
	borderline west = {0.5mm}{0.0mm}{black},
	fonttitle = \bfseries\sffamily,
	breakable,
	before skip = 3ex,
	after skip = 3ex
}{problem}

\tcbuselibrary{skins, breakable}

% --- You can define your own color box. Just copy the previous \newtcbtheorm definition and use the colors of yout liking and the title you want to use.
% --- Basic commands ---
%   Euler's constant
\newcommand{\eu}{\mathrm{e}}

%   Imaginary unit
\newcommand{\im}{\mathrm{i}}

%   Sexagesimal degree symbol
\newcommand{\grado}{\,^{\circ}}

% --- Comandos para álgebra lineal ---
% Matrix transpose
\newcommand{\transpose}[1]{{#1}^{\mathsf{T}}}

%%% Comandos para cálculo
%   Definite integral from -\infty to +\infty
\newcommand{\Int}{\int\limits_{-\infty}^{\infty}}

%   Indefinite integral
\newcommand{\rint}[2]{\int{#1}\dd{#2}}

%  Definite integral
\newcommand{\Rint}[4]{\int\limits_{#1}^{#2}{#3}\dd{#4}}

%   Dot product symbol (use the command \bigcdot)
\makeatletter
\newcommand*\bigcdot{\mathpalette\bigcdot@{.5}}
\newcommand*\bigcdot@[2]{\mathbin{\vcenter{\hbox{\scalebox{#2}{$\m@th#1\bullet$}}}}}
\makeatother

%   Hamiltonian
\newcommand{\Ham}{\hat{\mathcal{H}}}

%   Trace
\renewcommand{\Tr}{\mathrm{Tr}}

% Christoffel symbol of the second kind
\newcommand{\christoffelsecond}[4]{\dfrac{1}{2}g^{#3 #4}(\partial_{#1} g_{#2 #4} + \partial_{#2} g_{#1 #4} - \partial_{#4} g_{#1 #2})}

% Riemann curvature tensor
\newcommand{\riemanncurvature}[5]{\partial_{#3} \Gamma_{#4 #2}^{#1} - \partial_{#4} \Gamma_{#3 #2}^{#1} + \Gamma_{#3 #5}^{#1} \Gamma_{#4 #2}^{#5} - \Gamma_{#4 #5}^{#1} \Gamma_{#3 #2}^{#5}}

% Covariant Riemann curvature tensor
\newcommand{\covariantriemanncurvature}[5]{g_{#1 #5} R^{#5}{}_{#2 #3 #4}}

% Ricci tensor
\newcommand{\riccitensor}[5]{g_{#1 #5} R^{#5}{}_{#2 #3 #4}}
\newtheorem{problem}{Problem}

\begin{document}
	
	\begin{Large}
		\textsf{\textbf{This is a great title}}
		
		This is an even greater subtitle
	\end{Large}
	
	\vspace{1ex}
	
	\textsf{\textbf{Student:}} \text{Adil Hydari}, \href{mailto:adil.hydari@rutgers.edu}{\texttt{adil.hydari@rutgers.edu}}\\
	\textsf{\textbf{Professor:}} \text{Jorge Ortiz}
	
	
	\vspace{2ex}
	
	\begin{problem}{Mutual Information in Multimodal Systems}{problem-label}
		\begin{enumerate}[(a)]
			\item Use the definition of the mutual information equation:
			\[
			I(X;Y) = H(X) - H(X \mid Y)
			\]
			to explain how mutual information quantifies the reduction in uncertainty about Y when you know X1 and X2.
			\begin{enumerate}[label = (\roman*)]
				\item Under the assumption that $X_1$ and $X_2$ are conditionally independent given $Y$, the joint mutual information \[ I(X_1,X_2,Y) \] is simply the sum of the individual mutual information terms I($X_1$;Y) and I($X_2$;$Y$). This result indicates that the information provided by both modalities $X_1$ (sensor data) and $X_2$ (audio data) about $Y$ (the event or outcome) can be understood as the sum of the information each modality provides individually about $Y$.
			\end{enumerate}
			
			\item 
		\end{enumerate}
	\end{problem}
	
	Notice that the partial derivative and integral are smaller when used in a sentence compared with when you're working in a math environment like \verb|\begin{equation} \end{equation}|. If you want to display the full size of such commands in a sentence, you must use the command \verb|\displaystyle{}|, like it's shown here:
	
	\begin{problem}{Your title}{problem-label-2}
		This is an example problem taken from \cite{Sakurai2020}:
		
		\begin{enumerate}[(a)]
			\item Prove the following
			\begin{enumerate}[label = (\roman*)]
				\item $\langle p' | x | \alpha \rangle = \im \hbar \displaystyle{\pdv{}{p'} }\langle p' | \alpha \rangle$.
				
				\item $\langle \beta | x | \alpha \rangle = \displaystyle{\int \dd{p'} \phi_{\beta}^{*} (p') \im \hbar \pdv{}{p'} \phi_{\alpha} (p')}$, 
				
				\vspace{1ex}
				
				where $\phi_{\alpha}(p') = \langle p' | \alpha \rangle$ and $\phi_{\beta}(p') = \langle p' | \beta \rangle$ are momentum-space wave functions.
			\end{enumerate}
			
			\item $\cdots$
		\end{enumerate}
	\end{problem}
	
	I use the package \texttt{physics} which provides a great variety of commands for common operations and symbols. For instance, instead of typing \verb|\dfrac{\partial x}{\partial t}|, the \texttt{physics} package provides the command \verb|\pdv{x}{t}| which gives the same result. I also defined my own commands, so you can take a look in the \texttt{commands.tex} file if you like. I'd also suggest to create a folder and work each problem in a separate \texttt{.tex} file. I already included such folder in the \texttt{Overleaf} template, but you won't see it if you download the \texttt{Github} template. 
	
	% =================================================
	
	% \newpage
	
	% \vfill
	
	\bibliographystyle{apalike}
	\bibliography{references}
	
\end{document}