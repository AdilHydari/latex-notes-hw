\input{preamble}
\input{format}
\input{commands}
\newtheorem{problem}{Problem}

\begin{document}
	
	\begin{Large}
		\textsf{\textbf{This is a great title}}
		
		This is an even greater subtitle
	\end{Large}
	
	\vspace{1ex}
	
	\textsf{\textbf{Student:}} \text{Adil Hydari}, \href{mailto:adil.hydari@rutgers.edu}{\texttt{adil.hydari@rutgers.edu}}\\
	\textsf{\textbf{Professor:}} \text{Jorge Ortiz}
	
	
	\vspace{2ex}
	
	\begin{problem}{Mutual Information in Multimodal Systems}{problem-label}
		\begin{enumerate}[(a)]
			\item Use the definition of the mutual information equation:
			\[
			I(X;Y) = H(X) - H(X \mid Y)
			\]
			to explain how mutual information quantifies the reduction in uncertainty about Y when you know X1 and X2.
			\begin{enumerate}[label = (\roman*)]
				\item Under the assumption that $X_1$ and $X_2$ are conditionally independent given $Y$, the joint mutual information \[ I(X_1,X_2,Y) \] is simply the sum of the individual mutual information terms I($X_1$;Y) and I($X_2$;$Y$). This result indicates that the information provided by both modalities $X_1$ (sensor data) and $X_2$ (audio data) about $Y$ (the event or outcome) can be understood as the sum of the information each modality provides individually about $Y$.
			\end{enumerate}
			
			\item 
		\end{enumerate}
	\end{problem}
	
	Notice that the partial derivative and integral are smaller when used in a sentence compared with when you're working in a math environment like \verb|\begin{equation} \end{equation}|. If you want to display the full size of such commands in a sentence, you must use the command \verb|\displaystyle{}|, like it's shown here:
	
	\begin{problem}{Your title}{problem-label-2}
		This is an example problem taken from \cite{Sakurai2020}:
		
		\begin{enumerate}[(a)]
			\item Prove the following
			\begin{enumerate}[label = (\roman*)]
				\item $\langle p' | x | \alpha \rangle = \im \hbar \displaystyle{\pdv{}{p'} }\langle p' | \alpha \rangle$.
				
				\item $\langle \beta | x | \alpha \rangle = \displaystyle{\int \dd{p'} \phi_{\beta}^{*} (p') \im \hbar \pdv{}{p'} \phi_{\alpha} (p')}$, 
				
				\vspace{1ex}
				
				where $\phi_{\alpha}(p') = \langle p' | \alpha \rangle$ and $\phi_{\beta}(p') = \langle p' | \beta \rangle$ are momentum-space wave functions.
			\end{enumerate}
			
			\item $\cdots$
		\end{enumerate}
	\end{problem}
	
	I use the package \texttt{physics} which provides a great variety of commands for common operations and symbols. For instance, instead of typing \verb|\dfrac{\partial x}{\partial t}|, the \texttt{physics} package provides the command \verb|\pdv{x}{t}| which gives the same result. I also defined my own commands, so you can take a look in the \texttt{commands.tex} file if you like. I'd also suggest to create a folder and work each problem in a separate \texttt{.tex} file. I already included such folder in the \texttt{Overleaf} template, but you won't see it if you download the \texttt{Github} template. 
	
	% =================================================
	
	% \newpage
	
	% \vfill
	
	\bibliographystyle{apalike}
	\bibliography{references}
	
\end{document}