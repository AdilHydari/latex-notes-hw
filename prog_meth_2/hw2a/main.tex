\documentclass[letterpaper, 11pt]{extarticle}
% \usepackage{fontspec}

% ==================================================

% document parameters
% \usepackage[spanish, mexico, es-lcroman]{babel}
\usepackage[english]{babel}
\usepackage[margin = 1in]{geometry}

% ==================================================

% Packages for math
\usepackage{mathrsfs}
\usepackage{amsfonts}
\usepackage{amsmath}
\usepackage{amsthm}
\usepackage{amssymb}
\usepackage{physics}
\usepackage{dsfont}
\usepackage{esint}

% ==================================================

% Packages for writing
\usepackage{enumerate}
\usepackage[shortlabels]{enumitem}
\usepackage{framed}
\usepackage{csquotes}

% ==================================================

% Miscellaneous packages
\usepackage{float}
\usepackage{tabularx}
\usepackage{xcolor}
\usepackage{multicol}
\usepackage{subcaption}
\usepackage{caption}
\captionsetup{format = hang, margin = 10pt, font = small, labelfont = bf}

% Citation
\usepackage[round, authoryear]{natbib}

% Hyperlinks setup
\usepackage{hyperref}
\definecolor{links}{rgb}{0.36,0.54,0.66}
\hypersetup{
	colorlinks = true,
	linkcolor = black,
	urlcolor = blue,
	citecolor = blue,
	filecolor = blue,
	pdfauthor = {Author},
	pdftitle = {Title},
	pdfsubject = {subject},
	pdfkeywords = {one, two},
	pdfproducer = {LaTeX},
	pdfcreator = {pdfLaTeX},
}
\usepackage{titlesec}
\usepackage[many]{tcolorbox}

% Adjust spacing after the chapter title
\titlespacing*{\chapter}{0cm}{-2.0cm}{0.50cm}
\titlespacing*{\section}{0cm}{0.50cm}{0.25cm}

% Indent 
\setlength{\parindent}{0pt}
\setlength{\parskip}{1ex}

% --- Theorems, lemma, corollary, postulate, definition ---
% \numberwithin{equation}{section}

\newtcbtheorem[]{problem}{Problem}%
{enhanced,
	colback = black!5, %white,
	colbacktitle = black!5,
	coltitle = black,
	boxrule = 0pt,
	frame hidden,
	borderline west = {0.5mm}{0.0mm}{black},
	fonttitle = \bfseries\sffamily,
	breakable,
	before skip = 3ex,
	after skip = 3ex
}{problem}

\tcbuselibrary{skins, breakable}

% --- You can define your own color box. Just copy the previous \newtcbtheorm definition and use the colors of yout liking and the title you want to use.
% --- Basic commands ---
%   Euler's constant
\newcommand{\eu}{\mathrm{e}}

%   Imaginary unit
\newcommand{\im}{\mathrm{i}}

%   Sexagesimal degree symbol
\newcommand{\grado}{\,^{\circ}}

% --- Comandos para álgebra lineal ---
% Matrix transpose
\newcommand{\transpose}[1]{{#1}^{\mathsf{T}}}

%%% Comandos para cálculo
%   Definite integral from -\infty to +\infty
\newcommand{\Int}{\int\limits_{-\infty}^{\infty}}

%   Indefinite integral
\newcommand{\rint}[2]{\int{#1}\dd{#2}}

%  Definite integral
\newcommand{\Rint}[4]{\int\limits_{#1}^{#2}{#3}\dd{#4}}

%   Dot product symbol (use the command \bigcdot)
\makeatletter
\newcommand*\bigcdot{\mathpalette\bigcdot@{.5}}
\newcommand*\bigcdot@[2]{\mathbin{\vcenter{\hbox{\scalebox{#2}{$\m@th#1\bullet$}}}}}
\makeatother

%   Hamiltonian
\newcommand{\Ham}{\hat{\mathcal{H}}}

%   Trace
\renewcommand{\Tr}{\mathrm{Tr}}

% Christoffel symbol of the second kind
\newcommand{\christoffelsecond}[4]{\dfrac{1}{2}g^{#3 #4}(\partial_{#1} g_{#2 #4} + \partial_{#2} g_{#1 #4} - \partial_{#4} g_{#1 #2})}

% Riemann curvature tensor
\newcommand{\riemanncurvature}[5]{\partial_{#3} \Gamma_{#4 #2}^{#1} - \partial_{#4} \Gamma_{#3 #2}^{#1} + \Gamma_{#3 #5}^{#1} \Gamma_{#4 #2}^{#5} - \Gamma_{#4 #5}^{#1} \Gamma_{#3 #2}^{#5}}

% Covariant Riemann curvature tensor
\newcommand{\covariantriemanncurvature}[5]{g_{#1 #5} R^{#5}{}_{#2 #3 #4}}

% Ricci tensor
\newcommand{\riccitensor}[5]{g_{#1 #5} R^{#5}{}_{#2 #3 #4}}

\begin{document}
	
	\begin{Large}
		\textsf{\textbf{Programming Methodology II}}
		
		Homework 2a
	\end{Large}
	
	\vspace{1ex}
	
	\textsf{\textbf{Student:}} \text{Adil Hydari}, \href{mailto:adil.hydari@rutgers.edu}{\texttt{adil.hydari@rutgers.edu}}\\
	\textsf{\textbf{Professor:}} \text{Yao Liu}
	
	
	\vspace{2ex}
	
\begin{problem}{}{Problem-1}
	\begin{enumerate}
		\item Prove or disprove the assertions below.
		\begin{enumerate}[label=(\alph*)]
			\item $2^{2n} = O(2^n)$
			
			To find if \(2^{2n}\) $=$ \(O(2^n)\), we analyze the growth rates of the functions.
			
			\textbf{Def. of O:}
			
			A function \(f(n)\) is \(O(g(n))\) if there exist positive constants \(c\) and \(n_0\) such that:
			\[
			0 \leq f(n) \leq c \cdot g(n) \quad \forall n \geq n_0
			\]
			
			Let \(f(n) = 2^{2n}\) and \(g(n) = 2^n\).
			
			\[
			2^{2n} = (2^2)^n = 4^n
			\]
			
			We need to find constants \(c > 0\) and \(n_0\) such that:
			\[
			4^n \leq c \cdot 2^n \quad \forall n \geq n_0
			\]
			
			Dividing both sides by \(2^n\):
			\[
			2^n \leq c \quad \forall n \geq n_0
			\]
			
			However, \(2^n\) is not bounded by anything, so it will continue to increase as \(n\) increases. This means that there is no constant \(c\) exists that satisfies the inequality.
			
			\textbf{Therefore:}
			
			\(2^{2n}\) is $\neq$ \(O(2^n)\).
			
		\item $\max(100n^2, 50n^3) = \Omega(n^3)$
		
		To find if \(\max(100n^2, 50n^3)\) = \(\Omega(n^3)\), we can use the Def. of Omega.
		
		\(f(n)\) is \(\Omega(g(n))\) if there exist positive constants \(c\) and \(n_0\) such that (from example):
		\[
		f(n) \geq c \cdot g(n) \quad \forall n \geq n_0 
		\]
		
		\(f(n) = \max(100n^2, 50n^3)\) and \(g(n) = n^3\).
		
		When \(n \geq 2\):
		\[
		50n^3 \geq 100n^2 
		\]
		
		Thus, \(n \geq 2\):
		\[
		\max(100n^2, 50n^3) = 50n^3
		\]
		
		If we choose \(c = 50\) and \(n_0 = 2\):
		\[
		50n^3 \geq 50n^3 \quad \forall n \geq 2
		\]
		
		\textbf{Therefore:}
		
		\(\max(100n^2, 50n^3)\) = \(\Omega(n^3)\).
			
		\end{enumerate}
	\end{enumerate}
\end{problem}
\begin{problem}{}{Problem-2}
	Consider the following functions:
	\[
	\begin{aligned}
		f_1(n) &= n, \\
		f_2(n) &= n \cdot \log n, \\
		f_3(n) &= 
		\begin{cases} 
			n \cdot \log n & \text{if } n < 10, \\
			n & \text{if } n \geq 10, 
		\end{cases} \\
		f_4(n) &= n^2, \\
		f_5(n) &= 2^n, \\
		f_6(n) &= 2^{2n}.
	\end{aligned}
	\]
	
	Please state if the following assertions are True or False.
	\begin{enumerate}[label=(\alph*)]
        \item $f_1 = O(f_2)$
\item $f_1 = O(f_3)$
\item $f_1 = \Theta(f_3)$
\item $f_2 = \Theta(f_3)$
\item $f_1 = O(f_4)$
\item $f_2 = O(f_3)$
\item $f_3 = O(f_1)$
\item $f_4 = O(f_2)$
\item $f_4 = O(f_5)$
\item $f_6 = O(f_5)$
	\end{enumerate}
	
	\vspace{0.5em} % Adds some vertical space for readability
	
	\textbf{Solutions:}
	
	\begin{enumerate}[label=(\alph*)]
		\item \textbf{\(f_1 = O(f_2)\) }
		
		To find if \(f_1(n) = O(f_2(n))\), we use the Def. of Big-O:
		
		A function \(f(n)\) = \(O(g(n))\) if there are positive constants \(c\) and \(n_0\) such that:
		\[
		0 \leq f(n) \leq c \cdot g(n) \quad \forall n \geq n_0
		\]
		
		\(f(n) = n\) and \(g(n) = n \log n\).
		
		Find constants \(c > 0\) and \(n_0\) such that:
		\[
		n \leq c \cdot n \log n \quad \forall n \geq n_0
		\]
		
		Dividing both sides by \(n\):
		\[
		1 \leq c \cdot \log n \quad \forall n \geq n_0
		\]
		
		If we choose \(c = 1\) and \(n_0 = 2\) (since the log of 2 is 1):
		
		\[
		1 \leq 1 \cdot \log n \quad \forall n \geq 2
		\]
		
		
		\textbf{Therefore:}
		
		\(f_1(n) = O(f_2(n))\) is True.
		
		\item \textbf{\(f_1 = O(f_3)\) }
		\[
		f_3(n) = 
		\begin{cases} 
			n \cdot \log n & \text{if } n < 10, \\
			n & \text{if } n \geq 10. 
		\end{cases}
		\]
		
		
		For \(n \geq 10\):
		\[
		f_3(n) = n
		\]
		Thus, \(f_1(n) = n \leq c \cdot n\) for \(c \geq 1\).
		
		For \(n < 10\):
		\[
		f_3(n) = n \cdot \log n
		\]
		Since \(n \cdot \log n \geq n\) for \(n \geq 2\):
		\[
		n \leq c \cdot n \log n
		\]
		Choose \(c = 1\), as \(n \leq n \log n\) for \(n \geq 2\). For \(n = 1\), \(\log 1 = 0\), but since \(f_3(1) = 0\), we can choose a larger \(c\).
		
		\textbf{Therefore:}
		
		\(f_1(n) = O(f_3(n))\) is True.
		
		\item \textbf{\(f_1 = \Theta(f_3)\)}
		
		To find \(f_1(n) = \Theta(f_3(n))\), both \(f_1(n) = O(f_3(n))\) and \(f_1(n) = \Omega(f_3(n))\) must be true.
		
		From part B, we have \(f_1(n) = O(f_3(n))\).
		
		Now, find if \(f_1(n) = \Omega(f_3(n))\):
		
		\(f(n)\) = \(\Omega(g(n))\) if there are positive constants \(c\) and \(n_0\) such that:
		\[
		f(n) \geq c \cdot g(n) \quad \forall n \geq n_0
		\]
		
		For \(n \geq 10\):
		\[
		f_3(n) = n \implies f_1(n) = n \geq c \cdot n \quad \forall n \geq n_0
		\]
		Choose \(c = 1\).
		
		For \(n < 10\):
		\[
		f_3(n) = n \cdot \log n
		\]
		\(f_1(n) = n\) and \(n \cdot \log n > n\) for \(n > 1\).
		
		Thus, there is no constant \(c > 1\) that satisfies \(n \geq c \cdot n \log n\) for \(n < 10\).
		
		\textbf{Therefore:}
		
		 \(f_1(n)\) $\neq$ \(\Omega(f_3(n))\), \(f_1(n) = \Theta(f_3(n))\) is False
		
		\item \textbf{\(f_2 = \Theta(f_3)\) }
		
		To find if \(f_2(n) = \Theta(f_3(n))\), both \(f_2(n) = O(f_3(n))\) and \(f_2(n) = \Omega(f_3(n))\) must be true.
		
		\textbf{1. \(f_2(n) = O(f_3(n))\):}
		
		For \(n \geq 10\):
		\[
		f_3(n) = n
		\]
		\[
		f_2(n) = n \log n \leq c \cdot n \quad \text{needs} \quad \log n \leq c
		\]
		However, \(\log n\) grows without bound as \(n \to \infty\), so no constant \(c\) exists for this.
		
		\textbf{2. \(f_2(n) = \Omega(f_3(n))\):}
		
		For \(n \geq 10\):
		\[
		f_3(n) = n
		\]
		\[
		f_2(n) = n \log n \geq c \cdot n \quad \text{for} \quad c = 1, \quad \forall n \geq 10
		\]
		
		\textbf{Therefore:}
		
		\(f_2(n)\) $\neq$ \(O(f_3(n))\). \(f_2(n) = \Theta(f_3(n))\) is False.
		
		\item \textbf{\(f_1 = O(f_4)\) }
		
		\(f_1(n) = O(f_4(n))\).
		
		Using the Big-O def.:
		
		Constants \(c > 0\) and \(n_0\) such that:
		\[
		n \leq c \cdot n^2 \quad \forall n \geq n_0
		\]
		
		Divide both sides by \(n\) :
		\[
		1 \leq c \cdot n \quad \forall n \geq n_0
		\]
		
		Choose \(c = 1\) and \(n_0 = 1\):
		\[
		1 \leq 1 \cdot n \quad \forall n \geq 1
		\]
		Which holds true for \(n \geq 1\).
		
		\textbf{Therefore:}
		
		\(f_1(n) = O(f_4(n))\) is True.
		
		 \item \textbf{\(f_2 = O(f_3)\)}
		
		\(f_2(n) = O(f_3(n))\).
		
		Using the Big-O def.:
		
		Constants \(c > 0\) and \(n_0\) such that:
		\[
		n \log n \leq c \cdot f_3(n) \quad \forall n \geq n_0
		\]
		
		For \(n < 10\):
		\[
		f_3(n) = n \log n
		\]
		Thus:
		\[
		n \log n \leq c \cdot n \log n \quad \text{is satisfied for any } c \geq 1
		\]
		
		For \(n \geq 10\):
		\[
		f_3(n) = n
		\]
		Thus:
		\[
		n \log n \leq c \cdot n \quad \Rightarrow \quad \log n \leq c \quad \forall n \geq 10
		\]
		\(\log n\) grows without bound as \(n\) increases. Therefore, no constant \(c\) exists that satisfies \(\log n \leq c\) for all \(n \geq n_0\).
		
		\textbf{Therefore:}
		
		There is no constant \(c\) that satisfies the Big-O condition. \(n \geq n_0\), \(f_2(n) = O(f_3(n))\) is False.
		
		\item \textbf{\(f_3 = O(f_1)\)}
		
		\(f_3(n) = O(f_1(n))\).
		
		Using the Big-O def.:
		
		Constants \(c > 0\) and \(n_0\) such that:
		\[
		f_3(n) \leq c \cdot f_1(n) \quad \forall n \geq n_0
		\]
		
		For \(n < 10\):
		\[
		f_3(n) = n \log n
		\]
		The maximum value occurs at \(n = 10\):
		\[
		f_3(10) = 10 \cdot \log 10 \approx 10 \cdot 1 = 10
		\]
		
		For \(n \geq 10\):
		\[
		f_3(n) = n
		\]
		Thus:
		\[
		n \leq c \cdot n \quad \text{for } c \geq 1
		\]
		
		\(c = 10\) and \(n_0 = 1\)
		\[
		f_3(n) \leq 10 \cdot n \quad \forall n \geq 1
		\]
		
		\textbf{Therefore:}
		
		\(f_3(n) = O(f_1(n))\) is True.
		
		\item \textbf{\(f_4 = O(f_2)\)}
		
		\(f_4(n) = O(f_2(n))\).
		
		Using the Big-O def.:
		
		Constants \(c > 0\) and \(n_0\) such that:
		\[
		n^2 \leq c \cdot n \log n \quad \forall n \geq n_0
		\]
		
		Divide both sides by \(n\):
		\[
		n \leq c \cdot \log n \quad \forall n \geq n_0
		\]
		
		As \(n\) increases, \(n\) grows much faster than \(\log n\). So:
		\[
		\lim_{n \to \infty} \frac{n}{\log n} = \infty
		\]
		This shows that \(n\) eventually exceeds any multiple of \(\log n\), making the inequality \(n \leq c \cdot \log n\) impossible to satisfy for a large \(n\).
		
		\textbf{Therefore:}
		
		No such constant \(c\) exists to satisfy the Big-O. \(n \geq n_0\), \(f_4(n) = O(f_2(n))\) is False.
		
		\item \textbf{\(f_4 = O(f_5)\)}
		
		\(f_4(n) = O(f_5(n))\).
		
		Using the Big-O def.:
		
		Constants \(c > 0\) and \(n_0\) such that:
		\[
		n^2 \leq c \cdot 2^n \quad \forall n \geq n_0
		\]
		
		For \(n \geq 5\):
		\[
		2^n \geq n^3 > n^2
		\]
		We can choose \(c = 1\) and \(n_0 = 5\):
		\[
		n^2 \leq 2^n \quad \forall n \geq 5
		\]
		
		\textbf{Therefore:}
		
		\(f_4(n) = O(f_5(n))\) is True.
		
		\item \textbf{\(f_6 = O(f_5)\)}
		
		\(f_6(n) = O(f_5(n))\).
		
		Using the Big-O def.:
		
		Constants \(c > 0\) and \(n_0\) such that:
		\[
		2^{2n} \leq c \cdot 2^n \quad \forall n \geq n_0
		\]
		
		Then:
		\[
		2^{2n} = (2^n)^2 = 4^n \quad \text{and} \quad 4^n \leq c \cdot 2^n \quad \Rightarrow \quad 2^n \leq c \quad \forall n \geq n_0
		\]
		
		As \(n\) increases, \(2^n\) grows exponentially without bound. No constant \(c\) exists that satisfies \(2^n \leq c\) for \(n \geq n_0\).
		
		\textbf{Therefore:}
		
		 \(n \geq n_0\), \(f_6(n) = O(f_5(n))\) is False.
		
	\end{enumerate}
\end{problem}

\begin{problem}{}{Problem-3}
	SELECTIONSORT is another algorithm for sorting numbers in an array \( A[1 : n] \):
	\begin{verbatim}
		SELECTIONSORT(A, n)
		1 for i =1 to n - 1
		2     minIndex = i
		3     for j = i + 1 to n
		4         if A[j] < A[minIndex]
		5             minIndex = j
		6     if minIndex \neq i
		7         swap A[i] with A[minIndex]
	\end{verbatim}
	
	\begin{enumerate}[label=(\alph*)]
		\item Show that the running time of SELECTIONSORT is \( O(n^2) \).
		
		Big-O Def. :
		
		A function \( f(n) \) is \( O(g(n)) \) if there exist positive constants \( c \) and \( n_0 \) such that:
		\[
		f(n) \leq c \cdot g(n) \quad \forall n \geq n_0
		\]
		
		Analyzing the Algo. :
		
		SELECTIONSORT consists of two nested loops:
		
		\begin{enumerate}
			\item The outer loop runs from \( i = 1 \) to \( n-1 \), executing \( n-1 \) times.
			\item The inner loop runs from \( j = i+1 \) to \( n \), executing \( n - i \) times for each \( i \).
		\end{enumerate}
		
		 The total \# of Comparisons is the sum of the inner loop executions:
		\[
		\sum_{i=1}^{n-1} (n - i) = \sum_{k=1}^{n-1} k = \frac{n(n-1)}{2}
		\]
		
		\textbf{Big-O Bound:}
		
		\[
		f(n) = \frac{n(n-1)}{2} \leq \frac{n^2}{2} \leq c \cdot n^2 \quad \text{for } c = \frac{1}{2} \text{ and } n_0 = 1
		\]
		
		\textbf{Therefore:}
		
		The running time of SELECTIONSORT is \( O(n^2) \). 
		
		\item Show that the running time of SELECTIONSORT is \( \Omega(n^2) \).
		
		Big-Omega Def. :
		
		Function \( f(n) \) = \( \Omega(g(n)) \) if there are positive constants \( c \) and \( n_0 \) such that:
		\[
		f(n) \geq c \cdot g(n) \quad \forall n \geq n_0
		\]
		
		Analyzing the Algo. :
		
		 SELECTIONSORT has two nested loops with a total of \( \frac{n(n-1)}{2} \) comparisons.
		
		Big-Omega Bound:
		
		\[
		f(n) = \frac{n(n-1)}{2} \geq \frac{n(n-1)}{2} \geq \frac{n^2}{4} \quad \text{for } n \geq 2
		\]

		
		Choose \( c = \frac{1}{4} \) and \( n_0 = 2 \):
		\[
		f(n) \geq \frac{1}{4} \cdot n^2 \quad \forall n \geq 2
		\]
		
		\textbf{Therefore:}
		
		The running time of SELECTIONSORT is \( \Omega(n^2) \). 
		
		\item Show that the running time of SELECTIONSORT is \( \Theta(n^2) \).
		
		Big-Theta Def. :
		
		Function \( f(n) \) = \( \Theta(g(n)) \) if it is both \( O(g(n)) \) and \( \Omega(g(n)) \). And from parts (a) and (b), we have:
		\[
		f(n) = O(n^2) \quad \text{and} \quad f(n) = \Omega(n^2)
		\]
		
		\textbf{Therefore:}
		
		The running time of SELECTIONSORT is \( \Theta(n^2) \). 
	\end{enumerate}
\end{problem}

\begin{problem}{}{Problem-4}
	\textbf{Is an array that is in sorted (increasing) order a min-heap? Please provide your reasoning.}
	
	To determine whether a sorted (increasing order) array satisfies the properties of a min-heap, we must recall the definition of a min-heap.
	
	\textbf{Def. of a Min-Heap:}
	
	For every node \( i \) other than the root node, the value of \( i \) $\ge$ parent.
	The parent of a node at \( i \) is at \( \left\lfloor \frac{i}{2} \right\rfloor \). The children of the node at \( i \) are at \( 2i \) and \( 2i + 1 \).
	
	For every parent node \( A[i] \), Greater than or Equal to \( A[2i] \) and \( A[2i + 1] \)
	
	\textbf{Therefore:}
	
	Since a sorted and increasing order array displays that every parent node is less than or equal to its child nodes, the array satisfies min-heap.
	
\end{problem}
\begin{problem}{}{Problem 5}

Please illustrate the process of MAX-HEAPIFY(A,1) on array \( A = [16, 73, 92, 66, 69, 91, 36, 38, 20, 13, 26, 74, 32, 15, 11] \) using the “binary tree view”.

\textbf{Initial Array:}
\[
A = [16, 73, 92, 66, 69, 91, 36, 38, 20, 13, 26, 74, 32, 15, 11]
\]

\textbf{Initial Binary Tree Representation:}

\begin{center}
	\Tree
	[.{16}
	[.{73}
	[.{66}
	[.{38}
	[.{20} ] 
	[.{13} ]
	]
	[.{26} ]
	]
	[.{69}
	[.{15} ]
	[.{11} ]
	]
	]
	[.{92}
	[.{91}
	[.{74} ]
	[.{32} ]
	]
	[.{36}
	[.{15} ]
	[.{11} ]
	]
	]
	]
\end{center}


Step 1: Compare \( A[1] = 16 \) with its children \( A[2] = 73 \) and \( A[3] = 92 \).

The largest value among the parent and children is \( 92 \) at index \( 3 \).
Swap \( 16 \) with \( 92 \).

\textbf{Array After Swap:}
\[
A = [92, 73, 16, 66, 69, 91, 36, 38, 20, 13, 26, 74, 32, 15, 11]
\]

\textbf{Binary Tree After Step 1:}

\begin{center}
	\Tree
	[.{92}
	[.{73}
	[.{66}
	[.{38}
	[.{20} ] 
	[.{13} ]
	]
	[.{26} ]
	]
	[.{69}
	[.{15} ]
	[.{11} ]
	]
	]
	[.{16}
	[.{91}
	[.{74} ]
	[.{32} ]
	]
	[.{36}
	[.{15} ]
	[.{11} ]
	]
	]
	]
\end{center}


Step 2: Heapify the Subtree at Index \( 3 \) (Value \( 16 \)).

Compare \( A[3] = 16 \) with its children \( A[6] = 91 \) and \( A[7] = 36 \).
Swap \( 16 \) with \( 91 \).

\textbf{Array After Swap:}
\[
A = [92, 73, 91, 66, 69, 16, 36, 38, 20, 13, 26, 74, 32, 15, 11]
\]

\textbf{Binary Tree After Step 2:}

\begin{center}
	\Tree
	[.{92}
	[.{73}
	[.{66}
	[.{38}
	[.{20} ] 
	[.{13} ]
	]
	[.{26} ]
	]
	[.{69}
	[.{15} ]
	[.{11} ]
	]
	]
	[.{91}
	[.{16}
	[.{74} ]
	[.{32} ]
	]
	[.{36}
	[.{15} ]
	[.{11} ]
	]
	]
	]
\end{center}


Step 3: Heapify the Subtree at Index \( 6 \) (Value \( 16 \)).

Compare \( A[6] = 16 \) with its children \( A[12] = 74 \) and \( A[13] = 32 \).
Swap \( 16 \) with \( 74 \).

\textbf{Final Array After All Swaps:}
\[
A = [92, 73, 91, 66, 69, 74, 36, 38, 20, 13, 26, 16, 32, 15, 11]
\]

\textbf{Final Binary Tree After Step 3:}

\begin{center}
	\Tree
	[.{92}
	[.{73}
	[.{66}
	[.{38}
	[.{20} ] 
	[.{13} ]
	]
	[.{26} ]
	]
	[.{69}
	[.{15} ]
	[.{11} ]
	]
	]
	[.{91}
	[.{74}
	[.{16} ]
	[.{32} ]
	]
	[.{36}
	[.{15} ]
	[.{11} ]
	]
	]
	]
\end{center}

The largest element \( 92 \) is at the root, and all parent nodes being greater than or equal to their respective children, the max-heap build is finished.

\end{problem}

\begin{problem}{}{Problem-6}
	\textbf{Problem 6. (10 points)} \\
	Please illustrate the process of BUILD-MAX-HEAP on array \( A = [7, 40, 65, 91, 42, 69, 97, 11, 9, 28, 25, 19, 35, 2, 48] \) by showing \( A \) after each iteration of the for loop (not the “binary tree view”).
	
	\begin{verbatim}
		BUILD-MAX-HEAP(A, n)
		1 A.heap_size = n
		2 for i =  [n/2] down to 1
		3     MAX-HEAPIFY(A, i)
	\end{verbatim}
	\[
	A = [7, 40, 65, 91, 42, 69, 97, 11, 9, 28, 25, 19, 35, 2, 48]
	\]
	
	\begin{enumerate}[label=(\roman*)]
		\item \textbf{Iteration \( i = 7 \):} \\
		\textbf{Action:}  \texttt{MAX-HEAPIFY(A, 7)}. \\
		
		\[
		\begin{aligned}
			\text{Parent} &= A[7] = 97 \\
			\text{Left Child} &= A[14] = 2 \\
			\text{Right Child} &= A[15] = 48 \\
		\end{aligned}
		\]
		Since \( 97 \geq 2 \) and \( 97 \geq 48 \), no swaps are needed.
		
		\textbf{Array After Iteration 7:}
		\[
		A = [7, 40, 65, 91, 42, 69, 97, 11, 9, 28, 25, 19, 35, 2, 48]
		\]
		
		\item \textbf{Iteration \( i = 6 \):} \\
		\textbf{Action:}  \texttt{MAX-HEAPIFY(A, 6)}. \\
		
		\[
		\begin{aligned}
			\text{Parent} &= A[6] = 69 \\
			\text{Left Child} &= A[12] = 19 \\
			\text{Right Child} &= A[13] = 35 \\
		\end{aligned}
		\]
		Since \( 69 \geq 19 \) and \( 69 \geq 35 \), no swaps are needed.
		
		\textbf{Array After Iteration 6:}
		\[
		A = [7, 40, 65, 91, 42, 69, 97, 11, 9, 28, 25, 19, 35, 2, 48]
		\]
		
		\item \textbf{Iteration \( i = 5 \):} \\
		\textbf{Action:}  \texttt{MAX-HEAPIFY(A, 5)}. \\
		
		\[
		\begin{aligned}
			\text{Parent} &= A[5] = 42 \\
			\text{Left Child} &= A[10] = 28 \\
			\text{Right Child} &= A[11] = 25 \\
		\end{aligned}
		\]
		Since \( 42 \geq 28 \) and \( 42 \geq 25 \), no swaps are needed.
		
		\textbf{Array After Iteration 5:}
		\[
		A = [7, 40, 65, 91, 42, 69, 97, 11, 9, 28, 25, 19, 35, 2, 48]
		\]
		
		\item \textbf{Iteration \( i = 4 \):} \\
		\textbf{Action:}  \texttt{MAX-HEAPIFY(A, 4)}. \\
		
		\[
		\begin{aligned}
			\text{Parent} &= A[4] = 91 \\
			\text{Left Child} &= A[8] = 11 \\
			\text{Right Child} &= A[9] = 9 \\
		\end{aligned}
		\]
		Since \( 91 \geq 11 \) and \( 91 \geq 9 \), no swaps are needed.
		
		\textbf{Array After Iteration 4:}
		\[
		A = [7, 40, 65, 91, 42, 69, 97, 11, 9, 28, 25, 19, 35, 2, 48]
		\]
		
		\item \textbf{Iteration \( i = 3 \):} \\
		\textbf{Action:}  \texttt{MAX-HEAPIFY(A, 3)}. \\
		
		\[
		\begin{aligned}
			\text{Parent} &= A[3] = 65 \\
			\text{Left Child} &= A[6] = 69 \\
			\text{Right Child} &= A[7] = 97 \\
		\end{aligned}
		\]
		The largest among \( 65, 69, 97 \) is \( 97 \) at index \( 7 \). Swap \( A[3] \) with \( A[7] \).
		
		\textbf{Array After Swap:}
		\[
		A = [7, 40, 97, 91, 42, 69, 65, 11, 9, 28, 25, 19, 35, 2, 48]
		\]
		
		\textbf{Heapify Subtree at \( i = 7 \):} \\
		\[
		\begin{aligned}
			\text{Parent} &= A[7] = 65 \\
			\text{Left Child} &= A[14] = 2 \\
			\text{Right Child} &= A[15] = 48 \\
		\end{aligned}
		\]
		Since \( 65 \geq 2 \) and \( 65 \geq 48 \), no further swaps are needed.
		
		\textbf{Array After Iteration 3:}
		\[
		A = [7, 40, 97, 91, 42, 69, 65, 11, 9, 28, 25, 19, 35, 2, 48]
		\]
		
		\item \textbf{Iteration \( i = 2 \):} \\
		\textbf{Action:}  \texttt{MAX-HEAPIFY(A, 2)}. \\
		
		\[
		\begin{aligned}
			\text{Parent} &= A[2] = 40 \\
			\text{Left Child} &= A[4] = 91 \\
			\text{Right Child} &= A[5] = 42 \\
		\end{aligned}
		\]
		The largest among \( 40, 91, 42 \) is \( 91 \) at index \( 4 \). Swap \( A[2] \) with \( A[4] \).
		
		\textbf{Array After Swap:}
		\[
		A = [7, 91, 97, 40, 42, 69, 65, 11, 9, 28, 25, 19, 35, 2, 48]
		\]
		
		\textbf{Heapify Subtree at \( i = 4 \):} \\
		\[
		\begin{aligned}
			\text{Parent} &= A[4] = 40 \\
			\text{Left Child} &= A[8] = 11 \\
			\text{Right Child} &= A[9] = 9 \\
		\end{aligned}
		\]
		Since \( 40 \geq 11 \) and \( 40 \geq 9 \), no further swaps are needed.
		
		\textbf{Array After Iteration 2:}
		\[
		A = [7, 91, 97, 40, 42, 69, 65, 11, 9, 28, 25, 19, 35, 2, 48]
		\]
		
		\item \textbf{Iteration \( i = 1 \):} \\
		\textbf{Action:}  \texttt{MAX-HEAPIFY(A, 1)}. \\
		\[
		\begin{aligned}
			\text{Parent} &= A[1] = 7 \\
			\text{Left Child} &= A[2] = 91 \\
			\text{Right Child} &= A[3] = 97 \\
		\end{aligned}
		\]
		The largest among \( 7, 91, 97 \) is \( 97 \) at index \( 3 \). Swap \( A[1] \) with \( A[3] \).
		
		\textbf{Array After Swap:}
		\[
		A = [97, 91, 7, 40, 42, 69, 65, 11, 9, 28, 25, 19, 35, 2, 48]
		\]
		
		\textbf{Heapify Subtree at \( i = 3 \):} \\
		\[
		\begin{aligned}
			\text{Parent} &= A[3] = 7 \\
			\text{Left Child} &= A[6] = 69 \\
			\text{Right Child} &= A[7] = 65 \\
		\end{aligned}
		\]
		The largest among \( 7, 69, 65 \) is \( 69 \) at index \( 6 \). Swap \( A[3] \) with \( A[6] \).
		
		\textbf{Array After Swap:}
		\[
		A = [97, 91, 69, 40, 42, 7, 65, 11, 9, 28, 25, 19, 35, 2, 48]
		\]
		
		\textbf{Heapify Subtree at \( i = 6 \):} \\
		\[
		\begin{aligned}
			\text{Parent} &= A[6] = 7 \\
			\text{Left Child} &= A[12] = 19 \\
			\text{Right Child} &= A[13] = 35 \\
		\end{aligned}
		\]
		The largest among \( 7, 19, 35 \) is \( 35 \) at index \( 13 \). Swap \( A[6] \) with \( A[13] \).
		
		\textbf{Array After Swap:}
		\[
		A = [97, 91, 69, 40, 42, 35, 65, 11, 9, 28, 25, 19, 7, 2, 48]
		\]
		
		\textbf{Heapify Subtree at \( i = 13 \):} \\
		\[
		\begin{aligned}
			\text{Parent} &= A[13] = 7 \\
			\text{Left Child} &= A[26] = \text{N/A} \\
			\text{Right Child} &= A[27] = \text{N/A} \\
		\end{aligned}
		\]
		No more children, no further swaps are needed.
	
	\textbf{Final Array After BUILD-MAX-HEAP:}
	\[
	A = [97, 91, 69, 40, 42, 35, 65, 11, 9, 28, 25, 19, 7, 2, 48]
	\]
\end{enumerate}
	
\end{problem}




\begin{problem}{}{Problem-7}
	\textbf{In the for loop of the BUILD-MAX-HEAP procedure above, index \( i \) decreases from \( \left\lfloor \frac{n}{2} \right\rfloor \) to \( 1 \). Can we modify it so that \( i \) increases from \( 1 \) to \( \left\lfloor \frac{n}{2} \right\rfloor \) while still ensuring a correct max heap is built? Please provide your explanation.}
	
	\textbf{Explanation:}
	
	The BUILD-MAX-HEAP is made to construct a max heap from an unordered array by making sure that each subtree satisfies the max-heap properties. The standard implementation processes the nodes in a "bottom-up" way, starting from the last non-leaf node and moving upwards to the root. 
	
	\begin{enumerate}
		\item \textbf{Heapify Deps. :}
		\begin{itemize}
			\item When heapifying a node, it is essential that its children are already max heaps. Processing nodes from bottom to the top makes sure that by the time a parent node is heapified, all of its subtrees have already been heapified. 
			\item If we were to process nodes from top to bottom, we would end up having to heapify parent nodes before the children. This means that we would get situations where the children are not yet max heaps.
		\end{itemize}
	
	\textbf{Conclusion:}
	
	Modifying the `BUILD-MAX-HEAP` procedure to iterate \( i \) from \( 1 \) to \( \left\lfloor \frac{n}{2} \right\rfloor \) would change the heapify order, leading to an incorrect max heap. Therefore, the loop must continue to decrement \( i \) from \( \left\lfloor \frac{n}{2} \right\rfloor \) down to \( 1 \) to make sure the max heap happens correctly.
\end{enumerate}
\end{problem}

\begin{problem}{}{Problem-8}
	\textbf{Problem 8. (10 points)} \\
	Using the example in the slides as a model, please illustrate the process of HEAPSORT with \( A = [88, 43, 76, 22, 25, 17, 23] \). Here, \texttt{BUILD-MAX-HEAP} has already been executed, and \( A \) is a max-heap.
	
	\textbf{Max heap array:}
	\[
	A = [88, 43, 76, 22, 25, 17, 23]
	\]
	
	\textbf{Algo. PsuedoCode:}
	\begin{verbatim}
		HEAPSORT(A, n)
		1 BUILD-MAX-HEAP(A, n)
		2 for i = n downto 2
		3     exchange A[1] with A[i]
		4     A.heap_size = A.heap_size - 1
		5     MAX-HEAPIFY(A, 1)
	\end{verbatim}
	
	\textbf{HEAPSORT:}
	
	\begin{enumerate}[label=(\roman*)]
		\item \textbf{Iteration 1 (\( i = 7 \)):}
		
		\textbf{Action:} Swap the root \( A[1] = 88 \) with the final element \( A[7] = 23 \).
		
		\[
		A = [23, 43, 76, 22, 25, 17, 88]
		\]
		
		\textbf{Heap Size Reduction:} \( \text{heap\_size} = 6 \)
		
		\textbf{Apply \texttt{MAX-HEAPIFY(A, 1)}:}
		
		- Parent: \( A[1] = 23 \)\\
		- Left Child: \( A[2] = 43 \)\\
		- Right Child: \( A[3] = 76 \)\\
		
		Largest Value: \( A[3] = 76 \)
		
		Action: Swap \( A[1] \) with \( A[3] \).
		
		\[
		A = [76, 43, 23, 22, 25, 17, 88]
		\]
		
		\textbf{Heap After Iteration 1:}
		\[
		A = [76, 43, 23, 22, 25, 17, 88]
		\]
		
		\item \textbf{Iteration 2 (\( i = 6 \)):}
		
		\textbf{Action:} Swap the root \( A[1] = 76 \) with the final element in the heap \( A[6] = 17 \).
		
		\[
		A = [17, 43, 23, 22, 25, 76, 88]
		\]
		
		\textbf{Heap Size Reduction:} \( \text{heap\_size} = 5 \)
		
		\textbf{Apply \texttt{MAX-HEAPIFY(A, 1)}:}
		
		- Parent: \( A[1] = 17 \)\\
		- Left Child: \( A[2] = 43 \)\\
		- Right Child: \( A[3] = 23 \)\\
		
		Largest Value: \( A[2] = 43 \)
		
		Action: Swap \( A[1] \) with \( A[2] \).
		
		\[
		A = [43, 17, 23, 22, 25, 76, 88]
		\]
		
		\textbf{Heap After Iteration 2:}
		\[
		A = [43, 17, 23, 22, 25, 76, 88]
		\]
		
		\item \textbf{Iteration 3 (\( i = 5 \)):}
		
		\textbf{Action:} Swap the root \( A[1] = 43 \) with the final element in the heap \( A[5] = 25 \).
		
		\[
		A = [25, 17, 23, 22, 43, 76, 88]
		\]
		
		\textbf{Heap Size Reduction:} \( \text{heap\_size} = 4 \)
		
		\textbf{Apply \texttt{MAX-HEAPIFY(A, 1)}:}
		
		- Parent: \( A[1] = 25 \)\\
		- Left Child: \( A[2] = 17 \)\\
		- Right Child: \( A[3] = 23 \)\\
		
		Largest Value: \( A[1] = 25 \) (No swap needed)
		
		\textbf{Heap After Iteration 3:}
		\[
		A = [25, 17, 23, 22, 43, 76, 88]
		\]
		
		\item \textbf{Iteration 4 (\( i = 4 \)):}
		
		\textbf{Action:} Swap the root \( A[1] = 25 \) with the final element in the heap \( A[4] = 22 \).
		
		\[
		A = [22, 17, 23, 25, 43, 76, 88]
		\]
		
		\textbf{Heap Size Reduction:} \( \text{heap\_size} = 3 \)
		
		\textbf{Apply \texttt{MAX-HEAPIFY(A, 1)}:}
		
		- Parent: \( A[1] = 22 \)\\
		- Left Child: \( A[2] = 17 \)\\
		- Right Child: \( A[3] = 23 \)\\
		
		Largest Value: \( A[3] = 23 \)
		
		Action: Swap \( A[1] \) with \( A[3] \).
		
		\[
		A = [23, 17, 22, 25, 43, 76, 88]
		\]
		
		\textbf{Heap After Iteration 4:}
		\[
		A = [23, 17, 22, 25, 43, 76, 88]
		\]
		
		\item \textbf{Iteration 5 (\( i = 3 \)):}
		
		\textbf{Action:} Swap the root \( A[1] = 23 \) with the final element in the heap \( A[3] = 22 \).
		
		\[
		A = [22, 17, 23, 25, 43, 76, 88]
		\]
		
		\textbf{Heap Size Reduction:} \( \text{heap\_size} = 2 \)
		
		\textbf{Apply \texttt{MAX-HEAPIFY(A, 1)}:}
		
		- Parent: \( A[1] = 22 \)\\
		- Left Child: \( A[2] = 17 \)\\
		- Right Child: \( A[3] = \text{N/A} \) (heap size is 2)\\
		
		Largest Value: \( A[1] = 22 \) (No swap)
		
		\textbf{Heap After Iteration 5:}
		\[
		A = [22, 17, 23, 25, 43, 76, 88]
		\]
		
		\item \textbf{Iteration 6 (\( i = 2 \)):}
		
		\textbf{Action:} Swap the root \( A[1] = 22 \) with the final element in the heap \( A[2] = 17 \).
		
		\[
		A = [17, 22, 23, 25, 43, 76, 88]
		\]
		
		\textbf{Heap Size Reduction:} \( \text{heap\_size} = 1 \)
		
		\textbf{Apply \texttt{MAX-HEAPIFY(A, 1)}:}
		
		- Parent: \( A[1] = 17 \)\\
		- Left Child: \( A[2] = \text{N/A} \) (heap size is 1)\\
		- Right Child: \( A[3] = \text{N/A} \)\\
		
		Largest Value: \( A[1] = 17 \) (No swap needed)
		
		\textbf{Heap After Iteration 6:}
		\[
		A = [17, 22, 23, 25, 43, 76, 88]
		\]
	\end{enumerate}
	
	\textbf{Final Sorted Array:}
	\[
	A = [17, 22, 23, 25, 43, 76, 88]
	\]
	
\end{problem}


\begin{problem}{}{Problem-9}
	\textbf{Problem 9. (10 points)} \\
	Suppose that the objects in a max-priority queue are just keys. Illustrate the operation of \texttt{MAX-HEAP-INSERT(A, 70)} on the heap \( A = [98, 59, 79, 40, 35, 18, 68, 6, 28, 2] \) using the “binary tree view”.
	
	\textbf{Given Max-Heap Array:}
	\[
	A = [98, 59, 79, 40, 35, 18, 68, 6, 28, 2]
	\]
	
	\textbf{Initial Binary Tree Representation:}
	
	\begin{center}
		\Tree
		[.{98}
		[.{59}
		[.{40}
		[.{6} ]
		[.{28} ]
		]
		[.{35}
		[.{2} ]
		]
		]
		[.{79}
		[.{18} ]
		[.{68} ]
		]
		]
	\end{center}

	
	\textbf{Function Call: \texttt{MAX-HEAP-INSERT(A, 70)}}
	
	\begin{enumerate}[label=(\roman*)]
		\item \textbf{Insert the New Key at the End of the Array:}
		
		- Increase the heap size by 1.\\
		- Insert \( 70 \) at index \( 11 \).\\
		
		\[
		A = [98, 59, 79, 40, 35, 18, 68, 6, 28, 2, 70]
		\]
		
		\textbf{Binary Tree After Insertion:}
		
		\begin{center}
			\Tree
			[.{98}
			[.{59}
			[.{40}
			[.{6} ]
			[.{28} ]
			]
			[.{35}
			[.{2} ]
			[.{70} ]
			]
			]
			[.{79}
			[.{18} ]
			[.{68} ]
			]
			]
		\end{center}

		
		\item \textbf{Move the Inserted Key from top to bottom:}
		
		- Step 1: Compare \( 70 \) with its parent.\\
		- Index of 70: \( 11 \)\\
		- Parent Index: \( \left\lfloor \frac{11}{2} \right\rfloor = 5 \)\\
		- Parent Key: \( A[5] = 35 \)\\
		- Comparison: \( 70 > 35 \)\\
		- Action: Swap \( 70 \) with \( 35 \).\\
		
		\[
		A = [98, 59, 79, 40, 70, 18, 68, 6, 28, 2, 35]
		\]
		
		\textbf{Binary Tree After First Swap:}
		
		\begin{center}
			\Tree
			[.{98}
			[.{59}
			[.{40}
			[.{6} ]
			[.{28} ]
			]
			[.{70}
			[.{2} ]
			[.{35} ]
			]
			]
			[.{79}
			[.{18} ]
			[.{68} ]
			]
			]
		\end{center}

		
		- Step 2: Compare \( 70 \) with its new parent.\\
		- Index of 70: \( 5 \)\\
		- Parent Index: \( \left\lfloor \frac{5}{2} \right\rfloor = 2 \)\\
		- Parent Key: \( A[2] = 59 \)\\
		- Comparison: \( 70 > 59 \)\\
		- Action: Swap \( 70 \) with \( 59 \).\\
		
		\[
		A = [98, 70, 79, 40, 59, 18, 68, 6, 28, 2, 35]
		\]
		
		\textbf{Binary Tree After Second Swap:}
		
		\begin{center}
			\Tree
			[.{98}
			[.{70}
			[.{40}
			[.{6} ]
			[.{28} ]
			]
			[.{59}
			[.{2} ]
			[.{35} ]
			]
			]
			[.{79}
			[.{18} ]
			[.{68} ]
			]
			]
		\end{center}

		
		- Step 3: Compare \( 70 \) with its new parent.\\
		- Index of 70: \( 2 \)\\
		- Parent Index: \( \left\lfloor \frac{2}{2} \right\rfloor = 1 \)\\
		- Parent Key: \( A[1] = 98 \)\\
		- Comparison: \( 70 < 98 \)\\
		- Action: No swap needed. Heapify complete.\\
	\end{enumerate}
	
	\textbf{Final Array After \texttt{MAX-HEAP-INSERT}:}
	\[
	A = [98, 70, 79, 40, 59, 18, 68, 6, 28, 2, 35]
	\]
	
	\textbf{Final Binary Tree Structure:}
	
	\begin{center}
		\Tree
		[.{98}
		[.{70}
		[.{40}
		[.{6} ]
		[.{28} ]
		]
		[.{59}
		[.{2} ]
		[.{35} ]
		]
		]
		[.{79}
		[.{18} ]
		[.{68} ]
		]
		]
	\end{center}
	
\textbf{The final max-heap array is:}
	\[
	A = [98, 70, 79, 40, 59, 18, 68, 6, 28, 2, 35]
	\]
	
\end{problem}

	
	
	% =================================================
	
	% \newpage
	
	% \vfill
	
	\bibliographystyle{apalike}
	
\end{document}