\documentclass[11pt]{article}
\newcommand{\tb}{\textbf}
\begin{document}
	\title{\textbf{Introduction to Probability and Statistics for Engineers and Scientists}}
	\maketitle
\section{Combinatorics (Section 1.2.3 Cardinality)}	
\subsection{Counting Methods}
 A way to efficiently count the amount of elements in a set; most of what the counting method is based on is the multiplication principle. 
\subsubsection{Example 2.1}
 Suppose that I want to purchase a tablet computer. I can choose either a large or a small screen; a 64GB, 128GB, or 256GB storage capacity, and a black or white cover. How many different options do I have?
 \\
\textbf{There are 12 possible options; the multiplication principle states that we can multiply 2 x 3 x2 = 12}
\subsubsection{Theorem 1.3}
Any subset of a countable set is countable. Any super-set of an uncountable set is uncountable.
\subsubsection{Theorem 1.4}
If $A1,A2...$ is a list of countable sets, then the set $\cup_i A_i=A1 \cup A2 \cup A3...$ is also countable.
\subsubsection{Theorem 1.5}
If $A$ and $B$ are countable, then $A x B$ is also countable.
	
	
\end{document}